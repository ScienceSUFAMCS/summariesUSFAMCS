\documentclass[a4paper, 12pt]{report}
\usepackage{cmap}
\usepackage{amssymb}
\usepackage{amsmath}
\usepackage{graphicx}
\usepackage{amsthm}
\usepackage{setspace}
\usepackage[T2A]{fontenc}
\usepackage[utf8]{inputenc}
\usepackage[normalem]{ulem}
\usepackage{mathtext} % русские буквы в формулах
\usepackage[left=2cm,right=2cm, top=2cm,bottom=2cm,bindingoffset=0cm]{geometry}
\usepackage[english,russian]{babel}
\usepackage[unicode]{hyperref}
\newenvironment{Proof} % имя окружения
{\par\noindent{$\blacklozenge$}} % команды для \begin
{\hfill$\scriptstyle\boxtimes$}
\newcommand{\Rm}{\mathbb{R}}
\newcommand{\Cm}{\mathbb{C}}
\newcommand{\I}{\mathbb{I}}
\newcommand{\N}{\mathbb{N}}
\title{\textbf{\Huge{Дифференциальные уравнения}}\\Конспект по 2 курсу 
	специальности «прикладная математика»\\(лектор А. В. Филипцов)}
\date{}
\begin{document}
	\maketitle
	\tableofcontents{}
	\newpage
	\chapter{Основные понятия. Простейшие дифференциальные уравнения.}
	\section{Основные понятия.}
	$\bullet$ \textit{\textbf{Обыкновенным дифференциальным уравнением} называется выражение вида $$F(t, x, x', \dots, x^{(n)}) = 0,\eqno (1.1.1)$$ где $F$ --- некоторая функция $(n+2)$-ух переменных, определенная в некоторой области, $t$ --- независимая переменная, $x = x(t)$ --- неизвестная функция независимой переменной, $x',\dots, x^{(n)}$ --- производные функции $x(t)$, причем переменная $t$ и функции $F$ и $x$ действительны.}\\\\
	$\bullet$ \textit{Порядок старшей производной, присутствующей в уравнении $(1.1.1)$, называется \textbf{поряком} уравнения.}\\\\
	$\bullet$ \textit{\textbf{Решением} уравнения $(1.1.1)$ называется функция, заданная и $n$ раз дифференцируемая на некотором промежутке (связном множестве) $\I \subseteq \Rm$ и обращающая уравнение $(1.1.1)$ в верное равенство.}\\\\
	$\bullet$ \textit{График решения дифференциального уравнения называется \textbf{интегральной кривой}.}\\\\
	Если функция $x(t):\mathbb{I} \rightarrow \mathbb{R}$ является решением уравнения $(1.1.1)$ на промежутке $\mathbb{I}$, то $\forall I_1 \subseteq \mathbb{I}$ функция $x_1(t):I_1 \rightarrow \mathbb{R}$ такая, что $x_1(t) = x(t)\quad \forall t \in I_1$, является решением на промежутке $I_1$. При этом функция $x_1(t)$ называется \textbf{сужением} функции $x(t)$ на промежутке $I_1$, а функция $x(t)$ называется \textbf{продолжением} функции $x_1(t)$ на промежутке $\mathbb{I}$.\\\\
	$\bullet$ \textit{Решение, которое нельзя продолжить, называется \textbf{непродолжаемым}, а промежуток, на котором оно определено, называется \textbf{максимальным интервалом существования}.}\\\\
	Каждое дифференциальное уравнение имеет бесконечно много решений.\\\\
	$\bullet$ \textit{Совокупность решений уравнения $(1.1.1)$ вида $x = \varphi(t, C_1,\dots,C_n)$, зависящая от $n$ существенно произвольных постоянных, называется \textbf{общим решением} уравнения $(1.1.1)$.}\\\\
	Существенные постоянные --- это постоянные, которые нельзя заменить на меньшее количество, не изменив совокупность решений, описанных общим решением.\\\\
	$\bullet$ \textit{Решение дифференциального уравнения, получающееся из общего решения при конкретных произвольных постоянных, называется \textbf{частным решением}.}\\\\
	Возможны случаи, когда уравнение имеет решение, не входящее в общее решение.\\\\
	$\bullet$ \textit{Совокупность всех решений дифференциального уравнения называется \textbf{полным решением}.}\\\\
	Часто на практике математическая модель, содержащая дифференциальное уравнение, содержит также некоторые дополнительные условия необходимые для выбора единственного решения, описывающего моделируемый процесс.\\\\
	$\bullet$ \textit{Дополнительные условия накладываемые на неизвестную функцию называются \textbf{начальными}, если они относятся к одному значению аргумента, и \textbf{граничными}, если относятся к разным значениям аргумента.}\\\\
	$\bullet$ \textit{Начальная задача вида $$\begin{cases} F(t, x, x', \dots, x^{(n)}) = 0,\\ x|_{t = t_0} = \xi_0, x'|_{t = t_0} = \xi_1, \dots, x^{(n-1)}|_{t = t_0} = \xi_{n-1};\end{cases}\quad t_0 \in \mathbb{I}$$ называется \textbf{задачей Коши}}.
	\section{Простейшие дифференциальные уравнения.}
	Пусть $D$ --- оператор дифференцирования, то есть $D:x\mapsto x'$. Тогда первую производную функции $x$ будем обозначать $Dx = x'$, вторую $D^2x = x''$ и так далее.\\\\
	$\bullet$ \textit{\textbf{Простейшим} называется дифференциальное уравнение вида $$D^nx = f(t),\ t\in \mathbb{I},\eqno(1.2.1)$$ где $\mathbb{I}$ --- некоторый промежуток в $\mathbb{R}$, $f(t)$ --- непрерывная в $\mathbb{I}$ функция.}
	\newtheorem*{1_2_1}{Теорема}\begin{1_2_1}Общее решение простейшего дифференциального уравнения первого порядка $$Dx = f(t),\ t \in \mathbb{I}\eqno(1.2.2)$$ имеет вид $$x(t) = \int f(t)dt = \int\limits_{t_0}^tf(\tau)d\tau + C,$$ где $t_0 \in \mathbb{I}$, а $C$ --- произвольная постоянная.
	\end{1_2_1}\begin{Proof}
		Доказательство теоремы следует из курса математического анализа.
	\end{Proof}\\\\
	Заметим, что общее решение содержит все решения дифференциального уравнения (1.2.2). Следовательно, полученное общее решение является \textbf{полным} решением.
	\newtheorem*{1_2_2}{Теорема}\begin{1_2_2}Задача Коши $Dx = f(t),\ x|_{t = t_0} = \xi_0$, где $t, t_0 \in \mathbb{I}$, имеет единственное решение $$x(t) = \int\limits_{t_0}^tf(\tau)d\tau + \xi_0.$$
	\end{1_2_2}\begin{Proof}
		Подставим начальное условие в общее решение: 
		$\xi_0 = x(t_0) = \int\limits_{t_0}^{t_0}f(\tau)d\tau + C \Rightarrow C = \xi_0$.
	\end{Proof}\\\\
	Решение простейшего дифференциального уравнения (1.2.1) можно свести к последовательному решению простейшего дифференциального уравнения первого порядка. Так как $D^nx = D(D^{n-1}x)$, то уравнение $n$-го порядка является уравнением первого порядка относительно функции $D^{n-1}x$. Следовательно, из первой теоремы получаем $$D^{n-1} x = \int\limits_{t_0}^tf(\tau_1)d\tau_1 + C_1.$$ Тогда аналогично $$D^{n-2}x = \int\limits_{t_0}^t\Big(\int\limits_{t_0}^{\tau_2}f(\tau_1)d\tau_1 + C_1\Big)d\tau_2 + C_2 =  \int\limits_{t_0}^t\Big(\int\limits_{t_0}^{\tau_2}f(\tau_1)d\tau_1\Big)d\tau_2 + C_1(t-t_0) + C_2.$$
	Заметим, что семейство функций $C_1(t-t_0) + C_2$ описывает множество всех многочленов первой степени. Следовательно, это множество не изменится, если заменить его на $\widetilde{C}_1 t + \widetilde{C}_0$. Тогда имеем $$D^{n-2}x = \int\limits_{t_0}^td\tau_2\int\limits_{t_0}^{\tau_2}f(\tau_1)d\tau_1 + \widetilde{C}_1 t + \widetilde{C}_0.$$ Продолжая рассуждения аналогичным образом, получим $$x(t) = \int\limits_{t_0}^td\tau_n\int\limits_{t_0}^{\tau_n}d\tau_{n-1}\ldots\int\limits_{t_0}^{\tau_2}f(\tau_1)d\tau_1 + \widetilde{C}_{n-1}t^{n-1} + \ldots + \widetilde{C}_1 t + \widetilde{C}_0.$$
	\newtheorem*{1_2_3}{Теорема}\begin{1_2_3}Общее решение (полное) уравнения $(1.2.1)$ имеет вид $$x(t) = \underbrace{\int\limits_{t_0}^{t}\dfrac{(t-\tau)^{n-1}}{(n-1)!}f(\tau)d\tau}_{y_1} + \underbrace{\sum\limits_{i=0}^{n-1}\widetilde{C}_it^i}_{y_2}.\eqno(1.2.3)$$
	\end{1_2_3}\begin{Proof}
		Доказательство можно провести двумя способами: по формуле производной от интеграла, зависящего от параметра, или путем сведения кратных интегралов к повторным. Мы рассмотрим первый способ. Формула $$\dfrac{\partial}{\partial t}\Big(\int\limits_{\alpha(t)}^{\beta(t)}f(t,\tau)d\tau\Big) = \beta'(t)f(t,\beta(t)) - \alpha'(t)f(t,\alpha(t)) + \int\limits_{\alpha(t)}^{\beta(t)}f'(t,\tau)d\tau$$ определяет производную от интеграла, зависящего от параметра. Тогда для нашего случая подставим вместо $\alpha(t)$ и $\beta(t)$ соответственно $t, t_0$ и получим $$\dfrac{\partial}{\partial t}\Big(\int\limits_{t_0}^{t}f(t,\tau)d\tau\Big) = f(t,t) + \int\limits_{t_0}^{t}f'(t,\tau)d\tau.$$ Отсюда получаем $D^ny_2 = 0$. Вычислим $D^ny_1$:\\
		$$Dy_1 = \dfrac{(t-t)^{n-1}}{(n-1)!}f(t) + \int\limits_{t_0}^{t}\dfrac{(n-1)(t-\tau)^{n-2}}{(n-1)!}f(\tau)d\tau = \int\limits_{t_0}^{t}\dfrac{(t-\tau)^{n-2}}{(n-2)!}f(\tau)d\tau;$$
		$$D^2y_1 = \int\limits_{t_0}^{t}\dfrac{(t-\tau)^{n-3}}{(n-3)!}f(\tau)d\tau;$$
		$$\dots$$
		$$D^ny_1 = f(t).$$
	\end{Proof}
	\newtheorem*{1_2_4}{Теорема}\begin{1_2_4} Решение задачи Коши $D^nx = f(t)$, $D^ix|_{t=t_0} = \xi_i$ ($D^0$ --- тождественное отображение, то есть $D^0x = x$), где $t, t_0 \in \mathbb{I}$, всегда существует, единственно и имеет вид $$x(t) = \int\limits_{t_0}^{t}\dfrac{(t-\tau)^{n-1}}{(n-1)!}f(\tau)d\tau + \sum\limits_{i=0}^{n-1}\dfrac{\xi_i}{i!}(t-t_0)^i.$$
	\end{1_2_4}\begin{Proof}
		Разложим многочлен в общем решении простейшего дифференциального уравнения (1.2.3) по степеням $t-t_0$, то есть представим в виде $$\widetilde{\widetilde{C}}_{n-1}(t-t_0)^{n-1} +\ldots + \widetilde{\widetilde{C}}_1(t-t_0) + \widetilde{\widetilde{C}}_0,$$ где $\widetilde{\widetilde{C}}_i$ --- произвольные постоянные, зависящие от $\widetilde{C}_i$.\\ Тогда $\xi_0 = x|_{t=t_0} = \widetilde{\widetilde{C}}_0$, $\xi_i = D^ix|_{t=t_0} = i!\cdot \widetilde{\widetilde{C}}_i\Rightarrow \widetilde{\widetilde{C}}_i = \dfrac{\xi_i}{i!}$.
	\end{Proof}\\\\
	$\bullet$ \textit{\textbf{Комплекснозначной функцией действительного переменного} называется функция вида $$h(t) = f(t) + i\cdot g(t),$$ где $f(t)$ и $g(t)$ --- действительные функции, определенные на некотором промежутке $\mathbb{I}\in\mathbb{R}$.}\\\\
	Производные и интегралы комплекснозначной функции определяются следующим образом: $$h'(t) = f'(t) + i\cdot g'(t);$$ $$\int\limits_{t_0}^th(\tau)d\tau = \int\limits_{t_0}^tf(\tau)d\tau + i\cdot \int\limits_{t_0}^tg(\tau)d\tau.$$
	Кроме того для таких функций справедливы свойства производных:\begin{enumerate}
		\item $(\alpha u)' = \alpha u'$;
		\item $(u+v)' = u' + v';$
		\item $(uv)' = u'v + uv'.$
	\end{enumerate}
	Рассмотрим комплексное простейшее дифференциальное уравнение $Dz = h(t)$, где $h(t) = f(t) + i\cdot g(t)$. Если комплекснозначаная функция $z(t) = x(t) + i\cdot y(t)$ является решением этого дифференциального уравнения, то подставим его в уравнение и получим $\begin{cases}Dx = f(t),\\Dy = g(t).\end{cases}$ Следовательно, общее решение имеет вид $$z(t) = \int\limits_{t_0}^tf(\tau)d\tau + C_1 + i\cdot \Big(\int\limits_{t_0}^tg(\tau)d\tau + C_2\Big) = \int\limits_{t_0}^th(\tau)d\tau + C,\quad C\in\mathbb{C}.$$
	Тогда решение задачи Коши $Dz = h(t)$, $z|_{t=t_0} = \xi_0$ сводится к решению двух задач Коши:\begin{center}
		$\begin{cases}
			Dx = Re (h(t)),\\
			x|_{t=t_0} = Re(\xi_0);
		\end{cases}$\qquad $\begin{cases}
			Dy = Im (h(t)),\\
			y|_{t=t_0} = \xi_0;
		\end{cases}$
	\end{center}
	Заметим, что любое дифференциальное уравнение можно рассматривать как комплексное простейшее дифференциальное уравнение, у которого мнимая часть неоднородности равна нулю. Следовательно, можем говорить о существовании комплекснозначных решений действительных уравнений.
	\chapter{Линейные стационарные уравнения.}
	\section{Линейные стационарные дифференциальные уравнения. Существование и единственность задачи Коши.}
	$\bullet$ \textit{\textbf{Линейным дифференциальным уравнением n-ого порядка} называется уравнение вида $$D^nx + a_{n-1}(t)D^{n-1}x + \ldots + a_1(t)Dx + a_0(t)D^0x = f(t),\quad t\in \mathbb{I}\subseteq\mathbb{R},\eqno(2.1.1)$$ где функции $a_i(t)$ и $f(t)$ непрерывны на промежутке $\mathbb{I}$.}\\\\
	$\bullet$ \textit{Если $f(t) = 0$, то уравнение называется \textbf{однородным}, в противном случае \textbf{неоднородным}.}\\\\
	$\bullet$ \textit{Если $a_i(t)$ является постоянным, то уравнение \textbf{стационарное}.}\\\\
	Далее рассматриваем только стационарные линейные дифференциальные уравнения.\\\\
	Обозначим через $L_n = D^n + a_{n-1}D^{n-1} + \ldots + a_1D + a_0$ оператор дифференцирования. Тогда уравнение (2.1.1) запишем в виде $$L_nx=f(t).\eqno(2.1.1)$$ Так как для любых дифференцируемых функций $\varphi_1(t)$ и $\varphi_2(t)$ \begin{enumerate}
		\item $D(\varphi_1 + \varphi_2) = D(\varphi_1) + D(\varphi_2)$,
		\item $D(\alpha \varphi_1) = \alpha D(\varphi_1)$,
	\end{enumerate} то оператор дифференцирования $D$ является линейным. Следовательно, оператор $L_n$ является результатом композиции суммы и произведения на действительное число линейных операторов, а значит $L_n$ --- \textbf{линейный оператор}.\\\\
	Кроме действительных стационарных уравнений будем также рассматривать комплексные стационарные уравнения вида $L_nz = h(t)$, где $L_n$ --- линейный стационарный оператор с комплексными коэффициентами, а $h(t)$ --- комплекснозначная функция.\\\\
	$\bullet$ \textit{Любой комплекснозначный линейный оператор $L_n$ можно представить в виде композиции (произведения) операторов вида $D - \lambda_0D^0$. Такое представление линейного оператора $L_n$ называется \textbf{факторизацией} оператора.}\\\\
	Для факторизации оператора $L_n$ построим многочлен $\delta(\lambda) = \lambda^n+a_{n-1}\lambda^{n-1} + \ldots + a_1\lambda + a_0$ с теми же коэффициентами, что и у оператора $L_n$. Найдем корни этого многочлена над полем $\mathbb{C}$.\\\\
	$\bullet$ \textit{При этом многочлен $\delta(\lambda)$ называется \textbf{характеристическим многочленом} для оператора $L_n$, а уравнение $\delta(\lambda) = 0$ --- \textbf{характеристическим уравнением} для оператора $L_n$.}\\\\
	Многочлен $\delta(\lambda)$ над полем $\mathbb{C}$ с учетом кратности имеет столько корней, какова его степень. Обозначим эти корни через $\lambda_1,\ldots, \lambda_n$. Тогда $\delta(\lambda) = (\lambda - \lambda_1)(\lambda-\lambda_2)\ldots(\lambda - \lambda_n)$ и, следовательно, $L_n = (D-\lambda_1D^0)(D-\lambda_2D^0)\ldots(D-\lambda_nD^0)$ --- \textbf{факторизация} линейного оператора $L_n$.
	\newtheorem*{2_1_1}{Лемма}\begin{2_1_1}Решение задачи Коши $Dz - \lambda_0z = h(t)$, $z|_{t=t_0} = \xi_i$, $t\in\mathbb{I}\subseteq\mathbb{R}$ для любой комплекснозначной функции $h(t)$ и для любых комплексных чисел $\xi_i$ и $\lambda_0$ всегда существует и единственно.
	\end{2_1_1}\begin{Proof}
		Домножим уравнение задачи Коши на ненулевую функцию $e^{-\lambda_0t}$ и получим $$e^{-\lambda_0t}Dz \underbrace{ - \lambda_0 e^{-\lambda_0t}}_{D(e^{-\lambda_0t})}z = h(t) e^{-\lambda_0t},$$
		$$D(e^{-\lambda_0t}z)=h(t)e^{-\lambda_0t}.$$ Полученное уравнение является простейшим относительно функции $e^{-\lambda_0t}z$, следовательно, его общее решение имеет вид $$e^{-\lambda_0t}z = \int\limits_{t_0}^te^{-\lambda_0\tau}h(\tau)d\tau + C.$$ Тогда общее решение исходного уравнения имеет вид $$z = \int\limits_{t_0}^te^{\lambda_0(t-\tau)}h(\tau)d\tau + Ce^{\lambda_0t}.$$ Чтобы найти решение задачи Коши, подставим в общее решение начальные условия и получим $\xi = z|_{t=t_0} = Ce^{\lambda_0t_0}\Rightarrow C = \xi e^{-\lambda_0t_0}\Rightarrow$ $$z = \int\limits_{t_0}^te^{\lambda_0(t-\tau)}h(\tau)d\tau + \xi e^{\lambda_0(t-t_0)}.$$
	\end{Proof}
	\newtheorem*{2_1_2}{Теорема}\begin{2_1_2} Решение задачи Коши $L_nz = h(t)$, $D^iz|_{t=t_0} = \xi_i$, $i=\overline{0,n-1}$, $t\in\mathbb{I}\subseteq\mathbb{R}$ для любой непрерывной комплекснозначной функции $h(t)$, для любых комплексных чисел $\xi_i$ и для любого комплексного линейного оператора $L_n$ всегда существует и единственно.
	\end{2_1_2}\begin{Proof}
	Факторизуем оператор $L_n$: $L_n = (D-\lambda_1D^0)(D-\lambda_2D^0)\ldots(D-\lambda_nD^0)$ и обозначим через $L_{n-1}$ следующий оператор: $L_{n-1} = (D-\lambda_2D^0)(D-\lambda_3D^0)\ldots(D-\lambda_nD^0)$. Тогда уравнение задачи Коши представимо в виде $$(D-\lambda_1D^0)(L_{n-1}z) = h(t).$$
	Если обозначим $z_1=L_{n-1}z$, то уравнение имеет вид $(D - \lambda_1D^0)z_1 = h(t)\Rightarrow z_1|_{t=t_0}=(L_{n-1}z)|_{t=t_0} = \mu_1$, где $\mu_1$ --- комплексное число, которое выражается через $\xi_i$.\\\\
	По лемме решение задачи Коши всегда существует и единственно. Следовательно, функция $z(t)$ является решением задачи Коши $L_{n-1}z = z_1$, $D^iz|_{t=t_0} = \xi_i$, $i=\overline{0,n-2}$. Продолжая рассуждения аналогичным образом еще $(n-1)$ раз, получим функцию, которая является решением исходной задачи Коши.
\end{Proof}
\newtheorem*{2_1_3}{Следствие}\begin{2_1_3} Действительное решение задачи Коши $L_nx = f(t)$, $D^ix|_{t=t_0} = \xi_i$, $i=\overline{0,n-1}$, $t\in\mathbb{I}\subseteq\mathbb{R}$ для любой непрерывной действительной функции $f(t)$, для любых действительных чисел $\xi_i$ и для любого действительного линейного оператора $L_n$ всегда существует и единственно.
	\end{2_1_3}\begin{Proof}
	Если рассматривать функцию $f(t)$ и числа $\xi_i$ как комплекснозначную функцию и комплексные числа с нулевой мнимой частью, то по теореме задача Коши имеет единственное комплексное решение $z(t) = x(t) + i\cdot y(t)$, где функции $x(t)$ и $y(t)$ действительные. Но линейный оператор $L_n$ имеет действительные коэффициенты $L_nz = L_nx + i\cdot L_ny$.\\\\
	Тогда, так как $z$ --- решение задачи Коши, подставим: $$\begin{cases}
		L_nx + i\cdot L_ny = f(t) + i\cdot 0,\\
		(D^ix + i\cdot D^iy)|_{t=t_0} = \xi_i + i\cdot 0;
	\end{cases}$$ приравняем действительные части и получим, что функция $x(t) = Re(z(t))$ является единственным действительным решением задачи Коши.
\end{Proof}
	\section{Структура множества решений линейного однородного стационарного уравнения.}
	Рассмотрим линейное стационарное однородное уравнение $$L_nx = 0,\quad t\in \mathbb{R}.\eqno(2.2.1)$$
	Так как оператор $L_n$ является линейным, то для любых двух решений $x(t)$ и $y(t)$ уравнения (2.2.1) и $\forall\alpha,\beta \in\mathbb{R}\quad L_n(\alpha x + \beta y) = \alpha\cdot L_n x + \beta\cdot L_n y=\alpha\cdot 0 + \beta\cdot 0 = 0$. Следовательно, функция $\alpha x + \beta y$ также является решением уравнения (2.2.1), а значит множество решений уравнения (2.2.1) является \textbf{линейным (векторным) пространством}.\\\\
	Покажем, что пространство конечномерно и имеет размерность $n$ (равную порядку уравнения).\\\\
	$\bullet$ \textit{Функции $\varphi_1(t),\ldots,\varphi_n(t)$ называются \textbf{линейно зависимыми}, если существуют числа $\alpha_1,\ldots, \alpha_n$ не обращающиеся в $0$ одновременно такие, что $\alpha_1\varphi_1(t) + \ldots + \alpha_n\varphi_n(t) = 0\ \forall t$. В противном случае функции называются \textbf{линейно независимыми}.}\\\\
	$\bullet$ \textit{Пусть $\varphi_1(t),\ldots,\varphi_n(t)$ --- это $(n-1)$ раз дифференцируемые функции. Определитель $$W(t) = \begin{vmatrix}
		\varphi_1(t) & \dots & \varphi_n(t)\\
		D\varphi_1(t) & \dots & D\varphi_n(t)\\
		\vdots & \ddots & \vdots\\
		D^{n-1}\varphi_1(t) & \dots & D^{n-1}\varphi_n(t)
	\end{vmatrix}$$ называется \textbf{определителем Вронского}, или \textbf{Вронскианом} этих функций.}
	\newtheorem*{2_2_1}{Теорема}\begin{2_2_1}
		Если функции $\varphi_1(t),\dots,\varphi_n(t)$ линейно зависимые, то Вронскиан этих функций $W(t) = 0\ \forall t$.
	\end{2_2_1} \begin{Proof}
	Пусть функции $\varphi_1(t),\ldots,\varphi_n(t)$ линейно зависимы. Тогда одна из этих функций линейно выражается через остальные. Пусть, например, это функция $\varphi_1(t)$, то есть $\varphi_1(t) = \beta_2\varphi_2(t) + \ldots + \beta_n\varphi_n(t)$. Тогда $D^i\varphi_1(t) = \beta_2D^i\varphi_2(t) + \ldots + \beta_nD^i\varphi_n(t)\quad \forall i = \overline{1,n-1}$.\\ Следовательно, первый столбец Вронскиана $W(t)$ линейно выражается через остальные столбцы и, следовательно, $W(t) =0$ $\forall t$.
\end{Proof}
\newtheorem*{2_2_2}{Теорема} \begin{2_2_2} Если $n$ решений $\varphi_1(t), \ldots, \varphi_n(t)$ уравнения $(2.2.1)$ линейно независимы, то Вронскиан этих функций $W(t) \ne 0\ \forall t$.
\end{2_2_2} \begin{Proof}
От противного. Пусть $\exists t_0$, для которого $W(t_0) = 0$. Тогда линейная однородная система алгебраических уравнений с матрицей $W(t_0)$ имеет ненулевое решение, то есть существует ненулевой столбец $X = \begin{pmatrix}
	\alpha_1 \\ \vdots \\ \alpha_n
\end{pmatrix}$ такой, что $W(t_0)X = 0$. Тогда $$\begin{cases}
\alpha_1\varphi_1(t_0) + \alpha_2\varphi_2(t_0) + \ldots + \alpha_n\varphi_n(t_0) = 0,\\
\alpha_1D\varphi_1(t_0) + \alpha_2D\varphi_2(t_0) + \ldots + \alpha_nD\varphi_n(t_0) = 0,\\
\dotfill\\
\alpha_1D^{n-1}\varphi_1(t_0) + \alpha_2D^{n-1}\varphi_2(t_0) + \ldots + \alpha_nD^{n-1}\varphi_n(t_0) = 0.
\end{cases}$$ Рассмотрим функцию $\varphi(t) = \alpha_1\varphi_1(t) + \ldots + \alpha_n\varphi_n(t)$. Так как функция $\varphi(t)$ является линейной комбинацией решений уравнения (2.2.1), то она также является решением этого уравнения. При этом $D^i\varphi(t)|_{t=t_0} = (\alpha_1D^i\varphi_1(t) + \ldots + \alpha_nD^i\varphi_n(t))|_{t=t_0} = 0\ \forall i = \overline{0, n-1}$.\\\\
Но функция тождественно равная нулю также является решением, удовлетворяющим тем же начальными условиям. И, следовательно, по теореме о существовании и единственности решения задачи Коши функция $\varphi(t)$ и функция тождественно равная нулю равны между собой, то есть $\varphi(t) = 0\ \forall t \Rightarrow \alpha_1\varphi_1(t) + \ldots + \alpha_n\varphi_n(t) = 0\ \forall t$. Тогда функции $\varphi_1(t),\ldots,\varphi_n(t)$ линейно зависимы, что противоречит условию теоремы.
\end{Proof}
\newtheorem*{2_2_3}{Теорема}\begin{2_2_3}
	Множество решений линейного стационарного однородного уравнения порядка $n$ является конечномерным линейным пространством размерности $n$.
\end{2_2_3}
\begin{Proof}
	Рассмотрим $n$ задач Коши:
		$$\begin{cases}
			L_nx = 0,\\
			x|_{t=t_0} = 1,\\
			Dx|_{t=t_0} = 0,\\
			\dotfill\\
			D^{n-1}x|_{t=t_0} = 0;
		\end{cases}\quad\begin{cases}
		L_nx = 0,\\
		x|_{t=t_0} = 0,\\
		Dx|_{t=t_0} = 1,\\
		\dotfill\\
		D^{n-1}x|_{t=t_0} = 0;
	\end{cases}\dots\dots\dots\quad\begin{cases}
	L_nx = 0,\\
	x|_{t=t_0} = 0,\\
	Dx|_{t=t_0} = 0,\\
	\dotfill\\
	D^{n-1}x|_{t=t_0} = 1;
\end{cases}\eqno(2.2.2)$$
По теореме о существовании и единственности решения задачи Коши каждая из этих задач Коши имеет единственное решение. Обозначим их через $\varphi_0(t),\ldots,\varphi_{n-1}(t)$. Заметим, что их Вронскиан при $t=t_0$ равен определителю единичной матрицы, то есть равен 1. Следовательно, эти функции линейно независимы.\\\\
Покажем, что любое решение уравнения (2.2.1) линейно выражается через эти функции. Пусть $\varphi(t)$ --- произвольное решение уравнения (2.2.1), и пусть $\varphi(t)|_{t=t_0} = \xi_0$, $D\varphi(t)|_{t=t_0} = ~\xi_1$, $\ldots$, $D^{n-1}\varphi(t)|_{t=t_0} = \xi_{n-1}$. Рассмотрим функцию $\psi(t) = \xi_0\varphi_0 + \ldots + \xi_{n-1}\varphi_{n-1}$. При этом $D^i\psi(t)|_{t=t_0} = (\xi_0D^i\varphi_0 + \ldots + \xi_{n-1}D^i\varphi_{n-1})|_{t=t_0} = \xi_i\ \forall i = \overline{0, n-1}$. Следовательно, функция $\psi(t)$ является решением задачи Коши с теми же начальными условиями, что и функция $\varphi(t)$. Тогда по теореме о существовании и единственности решения задачи Коши $$\varphi(t) = \psi(t) = \xi_0\varphi_0 + \ldots + \xi_{n-1}\varphi_{n-1}.\eqno(2.2.3)$$ И, следовательно, произвольная функция $\varphi(t)$ линейно выражается через линейно независимые функции $\varphi_0(t),\ldots,\varphi_{n-1}(t)$. Значит функции $\varphi_0(t),\ldots,\varphi_{n-1}(t)$ составляют базис пространства решений.
\end{Proof}\\\\
$\bullet$ \textit{Базис пространства решений линейной однородной системы уравнений называется \textbf{фундаментальной системой решений}}.\\\\
$\bullet$ \textit{Фундаментальная система решений, удовлетворяющая условиям $(2.2.2)$ называется \textbf{нормированной} при $t = t_0$.}\\\\
Используя фундаментальную систему решений нормированную при $t=t_0$, задача Коши с произвольными начальными условиями может быть найдена по формуле (2.2.3).\\\\
$\bullet$ \textit{Функция $\psi(t)$ называется \textbf{сдвигом} функции $\varphi(t)$ на $t=t_0$, если $\psi(t) = \varphi(t-t_0)$.}
\newtheorem*{2_2_4}{Теорема о сдвиге}\begin{2_2_4}
	Если $\varphi(t)$ является решением задачи Коши $L_nx = 0$, $D^ix|_{t=0} = \xi_i$, $\forall i = \overline{0,n-1}$, то ее сдвиг $\psi(t) = \varphi(t-t_0)$ является решением задачи Коши  $L_nx = 0$, $D^ix|_{t=t_0} = \xi_i$, $\forall i = \overline{0,n-1}$.
\end{2_2_4}\begin{Proof}
	Так как $\varphi(t)$ --- решение уравнения (2.2.1), то $L_n\varphi(t) = 0\ \forall t$, то есть $$\dfrac{d^n\varphi(t)}{dt^n}+a_{n-1}\dfrac{d^{n-1}\varphi(t)}{dt^{n-1}} + \ldots + a_1\dfrac{d\varphi(t)}{dt} + a_0 \varphi(t) = 0\ \forall t.$$
	Так как равенство верно $\forall t$, оно останется верным, если заменить в нем $t$ на $t-t_0$, то есть
	$$\dfrac{d^n\varphi(t-t_0)}{d(t-t_0)^n}+a_{n-1}\dfrac{d^{n-1}\varphi(t-t_0)}{d(t-t_0)^{n-1}} + \ldots + a_1\dfrac{d\varphi(t-t_0)}{d(t-t_0)} + a_0 \varphi(t-t_0) = 0.$$ Заметим, что $$\dfrac{d\varphi(t-t_0)}{d(t-t_0)} = \dfrac{d\varphi(t-t_0)}{dt} \cdot \dfrac{1}{\frac{d(t-t_0)}{dt}} = \dfrac{d\varphi(t-t_0)}{dt} = \dfrac{d\psi(t)}{dt} = D\psi(t)\Rightarrow\dfrac{d^i\varphi(t-t_0)}{d(t-t_0)^i} = D^i\psi(t).$$ И, следовательно, $D^n\psi(t) + a_{n-1}D^{n-1}\psi(t) + \ldots + a_1D\psi(t) + a_0\psi(t) = 0$. То есть $\psi(t)$ также является решением уравнения (2.2.1) и при этом $D^i\psi(t)|_{t=t_0} = D^i\varphi(t-t_0)|_{t=t_0} =~D^i\varphi(t)|_{t=0} =~\xi_i.$~\end{Proof}\\\\
\textbf{Следствие.} \textit{Если $\varphi_0(t),\ldots,\varphi_{n-1}(t)$ --- фундаментальная система решений нормированная при $t=0$, то $\varphi_0(t-t_0),\ldots,\varphi_{n-1}(t-t_0)$ --- фундаментальная система решений нормированная при $t=t_0$.}\\\\
Следовательно, решение задачи Коши $L_nx = 0$, $D^ix|_{t=t_0} = \xi_i\ \forall i = \overline{0,n-1}$ имеет вид $x(t) = \xi_0\varphi_0(t-t_0) + \ldots + \xi_{n-1}\varphi_{n-1}(t-t_0)$.
\section{Построение фундаментальной системы решений линейного стационарного однородного уравнения.}
Рассмотрим уравнение $$L_nx = 0,\ t\in \Rm.\eqno(2.3.1)$$
И пусть $\Delta(\lambda)$ --- характеристический многочлен оператора $L_n$, имеющего вид $L_n = D^n + a_{n-1}D^{n-1} + \ldots + a_1D + a_0D^0$.
\newtheorem*{2_3_1}{Теорема}\begin{2_3_1}
	Если $\lambda_1,\ldots,\lambda_s$ --- корни над полем $\Cm$ характеристического многочлена $\Delta(\lambda)$ кратности $k_1,\ldots,k_s$ соответственно, то совокупность функций $$e^{\lambda_it}, te^{\lambda_it}, t^2e^{\lambda_it},\ldots,t^{k_i-1}e^{\lambda_it},\ i=\overline{1,s}\eqno(2.3.2)$$ является системой линейно независимых решений уравнения $(2.3.1)$.
\end{2_3_1}\begin{Proof}
Покажем, что функции совокупности (2.3.2) являются решениями уравнения (2.3.1). Так как $\lambda_i$ --- корень характеристического многочлена $\Delta(\lambda)$ кратности $k_i$, то линейный оператор $L_n$ при факторизации представим в виде $$L_n = L_{n-k_i}(D-\lambda_iD^0)^{k_i},$$ где $L_{n-k_i}$ --- дифференциальный оператор $(n-k_i)$-ого порядка.\\
Следовательно, функция, являющаяся решением дифференциального уравнения $L_n = L_{n-k_i}\underbrace{(D-\lambda_iD^0)^{k_i}x}_{=0}$, является решением уравнения (2.3.1).\\
Подействуем оператором $(D-\lambda_iD^0)^{k_i}$ на функцию $t^me^{\lambda_it}$, где $m < k_i$:\\\\
$(D-\lambda_iD^0)^{k_i}(t^me^{\lambda_it}) = (D-\lambda_iD^0)^{k_i-1}(D-\lambda_iD^0)(t^me^{\lambda_it}) = (D-\lambda_iD^0)^{k_i-1}(mt^{m-1}e^{\lambda_it} + \lambda_it^m e^{\lambda_it} - \lambda_it^m e^{\lambda_it}) = m(D-\lambda_iD^0)^{k_i-1}(t^{m-1}e^{\lambda_it}) = m(m-1)(D-\lambda_iD^0)^{k_i-2}(t^{m-2}e^{\lambda_it}) = \ldots = m(m-1)\cdot\ldots\cdot2\cdot1(D-\lambda_iD^0)^{k_i-m}(e^{\lambda_it}) = m!(D-\lambda_iD^0)^{k_i-m-1}\underbrace{(\lambda_ie^{\lambda_it} - \lambda_ie^{\lambda_it})}_{=0} = 0$.\\\\
Следовательно, функции $t^me^{\lambda_it}$ являются решениями уравнения (2.3.1) при $m < k_i$.\\\\
Покажем, что функции совокупности (2.3.2) линейно независимые. От противного. Пусть функции линейно зависимые, тогда существует нетривиальная линейная комбинация равная нулю. Она имеет вид $$P_1(t)e^{\lambda_1t} + P_2(t)e^{\lambda_2t} + \ldots + P_s(t)e^{\lambda_st} = 0,$$ где $P_i(t)$ --- многочлен степени не выше $k_i-1$.\\
Пусть $P_m(t)$, $m<s$ --- последний ненулевой многочлен из многочленов $P_i(t)$. Тогда полученная линейная комбинация имеет вид $$P_1(t)e^{\lambda_1t} + P_2(t)e^{\lambda_2t} + \ldots + P_m(t)e^{\lambda_mt} = 0,\ P_m(t)\ne 0.\eqno (2.3.3)$$ Домножим равенство (2.3.3) на функцию $e^{-\lambda_1t}$ и получим
$$P_1(t) + P_2(t)e^{(\lambda_2-\lambda_1)t} + \ldots + P_m(t)e^{(\lambda_m - \lambda_1)t} = 0.$$
Затем продифференцируем полуенное равенство $k_1$ раз:\\
$D^{k_1}P_1(t) = 0;$\\
$D^{k_1}(P_2(t)e^{(\lambda_2-\lambda_1)t}) = D^{k_1-1}(DP_2(t)e^{(\lambda_2-\lambda_1)t} + (\lambda_2-\lambda_1)P_2(t)e^{(\lambda_2-\lambda_1)t}) =\\ =D^{k_1-1}(\underbrace{(DP_2(t) + (\lambda_2-\lambda_1)P_2(t))}_{Q_2(t)}e^{(\lambda_2-\lambda_1)t}) = D^{k_1-1}(Q_2(t)e^{(\lambda_2-\lambda_1)t})=D^{k_1-2}(\widetilde{Q}_2(t)e^{(\lambda_2-\lambda_1)t}) =\\ = \ldots = \widetilde{\widetilde{Q}}_2(t)e^{(\lambda_2-\lambda_1)t}$, где $Q_2(t)$, $\widetilde{Q}_2(t)$, $\widetilde{\widetilde{Q}}_2(t)$ --- многочлены той же степени, что и $P_1(t)$. Тогда $$\widetilde{\widetilde{Q}}_2(t)e^{(\lambda_2 - \lambda_1)t} + \ldots + \widetilde{\widetilde{Q}}_m(t)e^{(\lambda_m - \lambda_1)t} = 0.$$ Проведем с полученным равенством ту же процедуру, что и с равенством (2.3.3), то есть домножим на $e^{-(\lambda_2-\lambda_1)t}$, а затем продифференцируем $k_2$ раз. В результате получим равенство вида $$\widetilde{\widetilde{R}}_3(t)e^{(\lambda_3 - \lambda_2)t} + \ldots + \widetilde{\widetilde{R}}_m(t)e^{(\lambda_m - \lambda_2)t} = 0,$$ где многочлены $\widetilde{\widetilde{R}}_i(t)$ имеют ту же степень, что и многочлен $P_i(t)$.\\
Продолжим эту процедуру $(m-1)$ раз. В результате получим выражение вида $$\widetilde{\widetilde{U}}_m(t)e^{(\lambda_m - \lambda_{m-1})t} = 0,$$ где многочлен $\widetilde{\widetilde{U}}_m(t)$ имеет ту же степень, что и многочлен $P_m(t)$.\\
Из последнего равенства следует, что $\widetilde{\widetilde{U}}_m(t) = 0$, но тогда и $P_m(t) = 0$, что противоречит выбору $P_m$. Значит функции линейно независимые. 
\end{Proof}
\newtheorem*{2_3_2}{Следствие}\begin{2_3_2}
	Если характеристический многочлен $\Delta(\lambda)$ над полем $\Cm$ имеет только действительные корни, то совокупность $(2.3.2)$ является фундаментальной системой решений уравнения $(2.3.1)$.
\end{2_3_2}\begin{Proof}
Если все $\lambda_i\in\Rm$, то (2.3.2) --- совокупность $n$ действительных линейно независимых решений уравнения (2.3.1).
\end{Proof}\\\\
Если среди корней $\lambda_i$ существует мнимый корень $\lambda = \alpha + \beta i$, то в совокупности (2.3.2) этому корню соответствуют комплекснозначные решения вида $t^me^{(\alpha + \beta i)t}$. А так как многочлен $\Delta(\lambda)$ имеет действительный коэффициент, то существует сопряженное число $\lambda = \alpha - \beta i$, также являющееся корнем характеристического уравнения, и, следовательно, в совокупности (2.3.2) будут содержаться решения вида $t^me^{(\alpha - \beta i)t}$.\\
Заменим в совокупности (2.3.2) пару функций на функции\begin{enumerate}
	\item $\dfrac{t^me^{(\alpha + \beta i)t} + t^me^{(\alpha - \beta i)t}}{2} = t^me^{\alpha t}cos\beta t = Re(t^me^{(\alpha + \beta i)t})$;
	\item $\dfrac{t^me^{(\alpha + \beta i)t} - t^me^{(\alpha - \beta i)t}}{2i} = t^me^{\alpha t}sin\beta t = Im(t^me^{(\alpha + \beta i)t})$.
\end{enumerate}
Заметим, что новые функции являются линейными комбинациями функций из совокупности (2.3.2), следовательно, они также являются решениями. При этом матрица перехода от исходной линейно независимой системы функций к новой системе имеет определитель равный произведению определителей вида $\begin{vmatrix}
	\frac{1}{2} & \frac{1}{2i}\\
	\frac{1}{2} & -\frac{1}{2i}
\end{vmatrix} = -\dfrac{1}{2i}\ne 0$. Следовательно, полученная система функций также линейно независимая, так как матрица перехода невырожденная.
\section{Неоднородные стационарные линейные уравнения. Метод Коши. Метод Лагранжа.}
Рассмотрим линейное неоднородное уравнение $$L_nx = f(t),\ t\in \mathbb{I}\eqno(2.4.1)$$ и соответствующее ему линейное однородное уравнение $$L_nx = 0.\eqno(2.4.2)$$
Пусть $f(t)$ --- непрерывная на $\mathbb{I}$ функция, и пусть оператор $L_n$ имеет вид $L_n = D^n + a_{n-1}D^{n-1} + \ldots + a_1D + a_0D^0$.\\\\
\textbf{\textit{Свойства решений линейных неоднородных уравнений:}}\begin{enumerate}
	\item \textit{Если $x_1$ --- решение уравнения $(2.4.1)$, то $\forall x_0$ решения уравнения $(2.4.2)$ функция $x_1 + x_0$ --- решение уравнения $(2.4.1)$.}
	\begin{Proof}
		$L_n(x_1 + x_0) = L_nx_1 + L_nx_0 = f(t) + 0 = f(t)$. Следовательно, функция $x_1 + x_0$ является решением.
	\end{Proof}
	\item \textit{Если $x_1$ --- решение уравнения $(2.4.1)$, то $\forall x_2$ решения уравнения $(2.4.1)$ функция $x_2 - x_1$ --- решение уравнения $(2.4.2)$.}
	\begin{Proof}
		$L_n(x_2-x_1) = L_nx_2 - L_nx_1 = f(t) - f(t) = 0$. Следовательно, функция $x_2 - x_1$ является решением.
	\end{Proof}
	\item \textbf{Принцип суперпозиции:} \textit{Если $x_1(t)$ --- решение уравнения $L_nx = f_1(t)$, а $x_2(t)$ --- решение уравнения $L_nx = f_2(t)$ с непрерывными функциями $f_1$ и $f_2$, то $x_1+x_2$ --- решение уравнения $L_nx = f_1(t) + f_2(t)$.}
\end{enumerate}
Из свойств 1 и 2 следует, что все решения неоднородного линейного уравнения (2.4.1) можно получить, если прибавить к частному решению неоднородного уравнения (2.4.1) все решения однородного уравнения (2.4.2). То есть $$x_{\text{OH}} = x_{\text{OO}} + x_{\text{ЧН}}.$$
\end{document}
