\documentclass[a4paper, 12pt]{report}
\usepackage[T2A]{fontenc} 
\usepackage[utf8]{inputenc}
\usepackage[english,russian]{babel} 
\usepackage{amsmath,amsfonts,amssymb,amsthm,mathtools, tipa}
\usepackage[left=2cm,right=2cm,top=2cm,bottom=2cm,bindingoffset=0cm]{geometry}
\usepackage{ upgreek, mathrsfs, cancel}
\usepackage[unicode]{hyperref}


\newcommand{\RomanNumeralCaps}[1]
    {\MakeUppercase{\romannumeral #1}}

\newenvironment{Proof} % имя окружения для доказательства
{\par\noindent{$\blacklozenge$}} % символ рядом с \begin
{\hfill$\scriptstyle\boxtimes$} % символ рядом с \end

\newenvironment{example} % имя окружения для примеров
{\par\noindent{\textsc{\textbf{Пример}.}}} % символ рядом с \begin
%{\hfill$\scriptstyle\Box$} % символ рядом с \end

\newenvironment{exercise}
{\par\noindent{\textsc{\textbf{Упражнение}.}}}
{\hfill}

\newtheorem*{theorem}{Теорема} % окружение для теорем
\newtheorem*{corollary}{Следствия} % окружение для следствий
\newtheorem*{lemma}{Лемма} % окружение для лемм

\newcommand{\Rm}{\mathbb{R}}
\newcommand{\Cm}{\mathbb{C}}
\newcommand{\I}{\mathbb{I}}
\newcommand{\N}{\mathbb{N}}
\newcommand{\Z}{\mathbb{Z}}
\newcommand{\Q}{\mathbb{Q}}
% команды для множеств 


\title{\textbf{\Huge{Дискретная математика и математическая логика}}\\Конспект по 2 семестру 
	специальностей «экономическая кибернетика» и «компьютерная безопасность»\\(лектор В. И. Бенедиктович)} % оформление титульного листа
\date{} % отключение даты в титульнике
\begin{document}
	\maketitle % отображение титульного листа в документе
	\tableofcontents{}
	\chapter{Булевы функции} % новая глава
	\section*{Замкнутые классы булевых функций}

Пусть $A \subseteq P$\\
$\bullet$ \textbf{Замыканием} $A$ называется множество функций из $P_2$, которые можно выразить в виде формул над $A$ и обозначается $[A]$ .\\\\
Свойства замыкания:
\begin{enumerate}
    \item $A \subseteq [A]$
    \item $A \subseteq B \Rightarrow [A] \subseteq [B]$
    \item $\big[[A]\big] = [A]$
    \item $[A] \cup [B] \subseteq [A \cup B]$
    \end{enumerate}
$\bullet$ 
$A$ - \textbf{полная система} булевой функции, если $[A] = P_2$.\\
$\bullet$ 
Система буевых функций $A$ \textbf{замкнутая}, если $[A] = A$.\\
\begin{example}
$A = \{1, x_1 \oplus x_2\}$ не замкнута, так как $1 \oplus 1 = 0 \notin A$\\
\end{example}

Пусть $A$ - замкнутый неполный класс системы булевых функций. Тогда если $A \subseteq B$, то $B$ - неполная система.\\
\begin{Proof}
$B \subseteq A \Rightarrow [B] \subseteq [A] \neq P_2 \Rightarrow [B] \neq P_2 \Rightarrow B$ - неполная система.
\end{Proof}\\
\begin{center}
    \textbf{Примеры замкнутых классов булевых функций}
    \end{center}
\RomanNumeralCaps{1}) Класс $T_0 = \{f(x_1, \dots , x_n) | f(0, \dots, 0) = 0\}$\\\\
Например:\\
$0,~ x, ~x_1 \cdot x_2, ~x_1 \vee x_2, ~x_1 \oplus x_2 \in T_0$\\
$1, ~\bar x, ~ x_1 \Rightarrow x_2, ~ x_1 | x_2, ~ x_1 \downarrow x_2, ~ x_1 \Leftrightarrow x_2 \notin T_0$\\\\
Мощность класса: $2^n-1$ ненулевых строк $\Rightarrow |T_0| = 2^2-1 = \frac12 2^{2^n}$
\begin{theorem}
Класс $T_0$ замкнут.
\end{theorem}
\begin{Proof}
    Поскольку $x \in T_0$, то достаточно показать, что если $f_1, f_2, \dots, f_n \in T_0$, то $f(f_1, \dots, f_n) \in T_0$. Действительно, $f(f_1(0, \dots, 0), \dots, f_n(0, \dots, 0)) = f(0, \dots, 0)=0$ 
\end{Proof}\\\\
\RomanNumeralCaps{2}) Класс $T_1 = \{f(x_1, \dots, x_n) \in P_2 | f(1, \dots, 1) = 1\}$\\\\
Например:\\
$1, x, x_1 \cdot x_2, x_1 \vee x_2, x_1 \Rightarrow x_2, x_1 \Leftrightarrow x_2 \in T_1$\\
$0, \bar x, x_1|x_2, x_1 \downarrow x_2, x_1 \oplus x_2 \notin T_1$
\begin{theorem}
Класс Класс $T_1$ замкнут.
\end{theorem}
\begin{Proof}
Доказательство аналогично доказательству предыдущей теоремы 
\end{Proof}\\\\
\RomanNumeralCaps{3}) Класс $M$ монотонных функций.\\\\
Введём \textbf{частичный булевый порядок} на $E^n_2$: $\bar \alpha = (\alpha_1, \alpha_2, \dots, \alpha_n)$, $\bar \beta = (\beta_1, \beta_2, \dots, \beta_n) \in E^n_2$\\
Говорят, что $\bar \alpha \leqslant \bar \beta \Leftrightarrow  \alpha_{i} \leqslant \beta_{i}$ для $\forall i$\\\\
$\bullet$ Функция $f(x_1, \dots, x_n)$ называется \textbf{монотонной}, если $\forall \bar \alpha, \bar \beta: \bar \alpha \leqslant \bar \beta \Rightarrow f(\bar \alpha) \leqslant f(\bar \beta)$\\
Множество всех монотонных функций обозначают $M$.\\\\
Например:\\
$0, 1, x, x_1 \cdot x_2, x_1 \vee x_2 \in M$\\
$0, \bar x, x_1 \Rightarrow x_2 \notin M$
\begin{theorem}
Класс $M$ замкнут.
\end{theorem}
\begin{Proof}
Достаточно показать, что если $f_1, f_2, \dots, f_m \in M$, то $f(f_1, \dots, f_m) \in M = \Phi$\\
Пусть $\bar \alpha \leqslant \bar \beta$ , тогда $f_1(\bar \alpha) \leqslant f_1(\bar \beta), \dots, f_m(\bar \alpha) \leqslant f_m(\bar \beta) \Rightarrow (f_1(\bar \alpha), \dots , f_m(\bar \alpha)) \leqslant (f_1(\bar \beta), \dots, f_m(\bar \alpha)) \Rightarrow f(f_1(\bar \alpha), \dots , f_m(\bar \alpha)) \leqslant f(f_1(\bar \beta), \dots, f_m(\bar \alpha))$, то есть $\Phi(\bar \alpha) \leqslant \Phi (\bar \beta)$
\end{Proof}\\
\begin{lemma}
О немонотонной функции\\
Если $f(x_1, \dots, x_n)$ - немонотонная функция, то $\bar x \in [\{0, 1, f\}]$ 
\end{lemma}
\begin{Proof}
Пусть $f(x_1, \dots, x_n)$ - немонотонная функция, то есть $\exists \bar \alpha < \bar \beta \Rightarrow f(\bar \alpha) = 1, f(\bar \beta) = 0 (1 > 0)$. $\bar \alpha < \bar \beta$ означает, что $\exists 1 \leqslant i_1 < i_2 < \dots < i_k \leqslant n$:\\
$\gamma_0 = \bar \alpha = (\alpha_1, \dots, \alpha_{i_1-1}, 0, \alpha_{i_1+1}, \dots, \alpha_{i_{2}-1}, 0, \alpha_{i_2 +1}, \dots, \alpha_{i_k -1}, 0, \alpha_{i_k +1},\dots,  \alpha_n)$  \\
$\gamma_1 = (\alpha_1, \dots, \alpha_{i_1-1}, 1, \alpha_{i_1+1}, \dots, \alpha_{i_{2}-1}, 0, \alpha_{i_2 +1}, \dots, \alpha_{i_k -1}, 0, \alpha_{i_k +1}, \dots,  \alpha_n)$  \\
$\gamma_2 = (\alpha_2, \dots, \alpha_{i_1-1}, 1, \alpha_{i_1+1}, \dots, \alpha_{i_{2}-1}, 1, \alpha_{i_2 +1}, \dots, \alpha_{i_k -1}, 0, \alpha_{i_k +1}, \dots,  \alpha_n)$  \\
$\dots$\\
$\gamma_k = \bar \beta = (\alpha_1, \dots, \alpha_{i_1-1}, 1, \alpha_{i_1+1}, \dots, \alpha_{i_{2}-1}, 1, \alpha_{i_2 +1}, \dots, \alpha_{i_k -1}, 1, \alpha_{i_k +1}, \dots,  \alpha_n)$  \\
$\gamma_0 < \gamma_1 < \gamma_2 < \gamma_3 < \dots < \gamma_k = \bar \beta$\\
Так как $f(\gamma_0) = 1, f(\gamma_k) = 0, f(\gamma_e) = 1, f(\gamma_{e+1}) = 0$, то $\exists l: 0 \leqslant l \leqslant k-1$, то есть $\alpha_e = 0, \beta_e = 1, \forall i \neq l,$  $\alpha_i = \beta_i$\\
Построим функцию $h(x) = f(\alpha_1, \dots, \alpha_{e-1}, x, \alpha_{e+1}, \dots, \alpha_n)$\\
$\begin{cases}
    h(0)  =  f(\bar \alpha) = 1 \\
    h(1) = f(\bar \beta) = 0
  \end{cases}$ $\Rightarrow h(x) \equiv \bar x$
\end{Proof}\\\\\\
\RomanNumeralCaps{4}) Класс $S$ самодвойственных функций.\\\\
$\bullet$ Функция $f^*(x_1, \dots, x_n) = \bar f( \bar x_1, \dots, \bar x_n)$ называется двойственной для функции $f(x_1, \dots, x_n)$\\
$\bullet$ Функция $f(x_1, \dots, x_n)$ называется самодвойственной, если $f(x_1, \dots, x_n) = f^*(x_1, \dots, x_n)$\\
Другими словами:\\
$$\bar f( x_1, \dots, x_n) = f( \bar x_1, \dots, \bar x_n) \eqno (1)$$
$\bullet$ Наборы $\bar \alpha = (\alpha_1, \dots, \alpha_n)$ и $\bar \beta = (\bar \alpha_1, \dots, \bar \alpha_n)$ называются противоположными наборами.\\\\
Например:\\
$x, \bar x \in S$\\
$x_1 \cdot x_2 \notin S$, то есть $(x_1 \cdot x_2)^* = \overline{\overline{x}_1 \cdot \overline{x}_2} = x_1 \vee x_2 \neq x_1 \cdot x_2$
\begin{theorem} 6 \quad
    Класс $S$ замкнут.
\end{theorem}
\begin{Proof}
   Достаточно показать, что $f_1, f_2, \dots, f_n \in S$, то $\Phi = f(f_1, \dots, f_n) \in S$\\
   $\Phi^* (x_1, \dots, x_n) = \bar \Phi (\bar x_1, \dots, \bar x_n) = \bar f(f_1(\bar x_1, \dots, \bar x_n), \dots, f_n(\bar x_1, \dots, \bar x_n))  \stackrel{(1)}{=}$ \\
   $\stackrel{(1)}{=} \bar f(\bar f_1(x_1, \dots, x_n), \dots, \bar f_n(x_1, \dots, x_n)) = f(f_1(x_1, \dots, x_n), \dots, f_n(x_1, \dots, x_n)) = \Phi(x_1, \dots, x_n)$
\end{Proof}
\begin{lemma}
    О несамодвойственной функции.\\
    Если $f(x_1, \dots, x_n)$ - несамодвойственная функция, то $0, 1 \in [\{\bar x, f\}]$
\end{lemma}
\begin{Proof}
    Пусть $f(x_1, \dots, x_n)$ - несамодвойственная функция. Тогда $\exists \bar \alpha = (\alpha_1, \dots, \alpha_n), f(\bar \alpha) = f(\bar \alpha_1, \dots, \bar \alpha_n) = f(\alpha_1, \dots, \alpha_n)$ .\\
    Заменим $\alpha_i$ на $x \oplus \alpha_i$: $\begin{cases}
        x, \text{если} ~ \alpha_i = 0,\\
        \bar x, \text{если} ~ \alpha_i = 1; 
    \end{cases}$\\
    Получим функцию $h(x) \equiv f(x \oplus \alpha_1, \dots, x \oplus \alpha_n)$\\
    $h(0) = f(\alpha_1, \dots, \alpha_n) = c, ~ c = const$\\
    $h(1) = f(\bar \alpha_1, \dots, \bar \alpha_n) = c$\\
    $h(x) = c \Rightarrow \bar c = \bar h(x) \Rightarrow 0, 1 \in [\{\bar x, f\}]$
\end{Proof}
\begin{center}
    \textbf{Полином Жегалкина}
    \end{center}
$\bullet$ Полином Жегалкина — функция вида $\sum\limits_{\{i_1, \dotso, i_k\}\in
\{1, 2, \dotso, n\}} 
a_{i_1,\dotso, a_k} \cdot x_{i_1} 
\cdot \dotso \cdot x_{i_k} \oplus a$ , где $a$ — свободный член.\\\\
Пример: $x_1x_2x_3 \oplus x_1x_2 \oplus x_1x_3 \oplus x_1 \oplus 1$\\
\begin{center}
    Полные системы булевых функций
    \end{center}
    \begin{theorem}7 \quad
    Система функций $A= {x_1\vee x_2, x_1\cdot x_2, \bar x}$ является полной. (Базис Буля)
        \end{theorem}
\begin{Proof}
$f(x_1,\dotso,x_n) \in P_2$. Если булевая функция отлична от нуля, то по следствию из 2 теоремы Шеннона функция $f$ выражается  в виде совершенной дизъюнктивной нормальной формы, в которую входят дизъюнкция, конъюнкция, отрицание, тем самым она принадлежит замыканию класса. Если $f = 0$, то $f = x_1\cdot \bar x_1$.    
\end{Proof}
\begin{theorem} 8 (о сведении)\\
  Если система $A$ — полная и любая функция из $A$ может быть выражена формулой над некоторой системой функций $B$, то $B$ — полная система.  
\end{theorem}
\begin{Proof}
    $[A]=P_2, A \subseteq [B] \Rightarrow P_2 = [A] \subseteq \big[[B]\big] = [B] \subseteq P_2 \Rightarrow [B] = P_2$. То есть $B$ - полная система.
\end{Proof}
\begin{theorem}9\\
 Сдедующие системы являются полными:
 \begin{enumerate}
        \item $A_1 = \{x_1 \vee x_2,~ \bar x\}$
	  \item $A_2 = \{x_1 \cdot x_2,~ \bar x\}$
	  \item $A_3 = \{x_1|x_2\}$
	  \item $A_4 = \{x_1 \cdot x_2, ~x_1\oplus x_2, ~1\}$
  \end{enumerate}
\end{theorem}
\begin{Proof}
\begin{enumerate}
    \item 	По теореме 7 система $\{x_1 \vee x_2,~ \bar x\}$ — полная. По закону де Моргана: $x_1\cdot x_2 = \overline{\overline{x}_1 \vee \overline{x}_2} \Rightarrow x_1\cdot x_2 \in [A_1]$. По теореме 8 следует$[A_1] = P_2$.
\item	По закону де Моргана $x_1 \vee x_2 = \overline{\overline{x}_1 \cdot \overline{x}_2} \Rightarrow x_1\vee x_2 \in [A_2]$. По теореме $8 [A_2] = P_2$.
\item	Можем представить отрицание в виде штриха Шеффера: $\bar x =x|x. x_1 \cdot x_2 = \overline{x_1|x_2} = (x_1|x_2) | (x_1|x_2) \Rightarrow \bar x, ~~ x_1\cdot x_2 \in [A_3]$. По теореме $8$ и доказательству п.$2$ $A_3 = P_2$.
\item	$\bar x = x\oplus 1 \Rightarrow \bar x = [A_4]$. По теореме $8$ и доказательству п.$2$ следует, что $[A_4] = P_2$.
\end{enumerate}
\end{Proof}	
\begin{theorem}$10$ (теорема Жегалкина)\\
Любую булевую функцию $f(x_1, \dotso, x_n)$ можно представить единственным образом в виде полинома Жегалкина $G_f(x_1, \dotso, x_n)$.
\end{theorem}
\begin{Proof}
1) Докажем существование:\\
В силу теоремы $9$ и доказательства п.$4$ система $\{x_1 \cdot x_2, ~x_1\oplus x_2, ~1\}$ полная $\Rightarrow$ любая булевая функция $f(x_1, \dotso, x_n)$ может быть реализована над этой системой. После раскрытия скобок используют дистрибутивность конъюнкции относительно сложения по mod $2$ ($\oplus$) и приведения подобных получаем полином Жегалкина.\\\\
2) Докажем единственность:\\
Подсчитаем количество полиномов Жегалкина от переменных $x_1, \dotso, x_n$. Каждое слагаемое в полиноме Жегалкина имеет вид конъюнкции переменных $x_{i_{1}},\dotso,x_{i_{k}}$ или существует свободный член $1$. Каждая такая конъюнкция определяется подмножеством индексов во множестве индексов $i = \overline{1, n}$ ($\{ i_1, \dotso, i_k\} \subset \{1, \dotso, n\}$). Значит, множество всевозможных слагаемых в полиноме равно количеству подмножеств в $n$-элементном множестве, то есть $2^n$.\\
Чтобы составить полином Жегалкина нужно выбрать подмножество из множества всевозможных слагаемых. Число полиномов Жегалкина равно ${2^2}^n$, что равно количеству булевых функций от $n$ переменных. А так как любая булевая функция имеет полином Жегалкин, представляющий её, то существует единственный полином представляющий булевую функцию.
\end{Proof}\\\\
В силу этой функции полином Жегалкина представляет собой булевую функцию $G_f$.  $G_f(x_1,\dotso,x_n)$ — алгебраически нормальная формула (АНФ) булевой функции. \\\\
Булевая функция $f(x_1,\dotso,x_n)$ существенно зависит от $x_i$ (не является фиктивной переменной) и содержится в каком-либо слагаемом $G_f(x_1,\dotso,x_n)$.\\\\
Пример: $x_1\vee x_2$
\begin{center}
	Методы приведения к виду полинома Жегалкина
 \end{center}
\begin{enumerate}
    \item Метод неопределенных коэффициентов\\ 
$x_1 \vee x_2 = a\cdot x_1x_2 + b\cdot x_1 + c\cdot x_2 +d$; нам необхожимо найти $a, b, c, d$.
Подставим  $(0, 0), (0, 1), (1, 0), (1, 1)$:\\\\
$(0,0)	~~~~~~ d=0$\\
$(0,1)	~~~~~~ 1=c+d \Rightarrow c=1$\\
$(1,0)	~~~~~~ 1=b+d \Rightarrow b=1$\\
$(1,1)	~~~~~~ 1=a+b+c+d \Rightarrow a=1$ (по mod $2$)\\\\
Следовательно, $x_1\vee x_2 = x_1x_2\oplus x_1\oplus x_2$.\\
В общем случае для определения неизвестных коэффициентов при $a_{i_1, \dotso, i_k} x_{i_1}, \dotso, x_{i_k}$ составляется уравнение $G_f(a_1, \dotso, a_i) = f(a_1, \dotso, a_n)$, из чего следует, что всего $2^n$ уравений, $2^n$ неизвестных коэффициентов и в силу теоремы $10$ имеет единственное решение.
\item	Метод эквивалентных преобразований\\
С помощью следующих тождеств: $\bar A=A\oplus 1, ~~ A\vee B = \overline{\overline{A} \cdot \overline{B}} = (A\oplus 1) \cdot (B\oplus 1) \oplus 1 = AB \oplus A \oplus B,~~ A\cdot A=A, ~~ A\cdot 1=A,~~ A\oplus A=0,~~ A\oplus 0=A$, приводим формулу к эквивалентной над системой этих трёх функций ${x_1\cdot x_2, x_1\oplus x_2, \bar x}$ и запишем в виде
 $x_1\vee x_2 = x_1\cdot x_2\oplus x_1\oplus x_2$.
\item Метод треугольника Паскаля \\
Используется, когда функция задана вектором значений. Метод позволяет преобразовать таблицу истинности в полином Жегалкина путем построения вспомогательной треугольной таблицы в соответствии со следующими правилами:
\begin{enumerate}
    \item Строится таблица истинности, в которой строки идут в лексикографическом порядке возрастания двоичных кодов (от $0$ до $1$): $00\dotso 00$, $00\dotso 01$, $00\dotso 10$, $00\dotso 11$, $\dotso$, $11\dotso 11$; 
    \item Строится вспомогательная треугольная таблица, в которой первый столбец совпадает со столбцом значений функции из таблицы истинности;
    \item Ячейка в каждом последующем столбце таблицы получается путем суммирования по mod $2$ двух ячеек: стоящей в той же строке и в строке ниже предыдущего столбца;
    \item Столбцы вспомогательной треугольной таблицы нумеруются двоичными кодами в том же порядке, что и строки таблицы истинности;
    \item Каждому двоичному коду ставится в соответствие один из членов полинома Жегалкина в зависимости от позиций кода, в которых стоят единицы;
    \item Если в верхней строке любого столбца стоит $1$, то соответствующий член входит в полином Жегалкина.
    \end{enumerate}
\begin{tabular}{|c|c|c|}
\hline
$x_1$ & $x_2$ & $x_1 \vee x_2$\\
\hline
   0  & 0 & 0 \\
   0  & 1 & 1\\
   1 & 0 & 1\\
   1 & 1 & 1\\
\hline
\end{tabular}


$\begin{tabular}{c|c|c|c}
 00 & 01 & 10 & 11 \\
\hline
$1$ & $x_1$ & $x_2$ & $x_1 \cdot x_2$\\
\hline
0 & 1 & 1 & 1 \\
1 & 0 & 0\\
1 & 0\\
1\\
\end{tabular}$\\
Из треугольника Паскаля результат: $x_1\vee x_2 = x_1\cdot x_2\oplus x_1\oplus x_2$
\end{enumerate}
V) Класс $L$ линейных функций.
\\\\
$\bullet$ Булевая функция $f(x_1,\dotso,x_n)$ \textbf{линейная}, если она может быть задана в виде полинома Жегалкина степени $\leqslant 1$.
	$$f(x_1,\dotso,x_n)=a_0 \oplus a_1x_1\oplus a_2x_2\oplus \dotso \oplus a_nx_n$$
 где $a_i \in E_2 = \{0, 1\},
 ~~ i= \overline{0, n}$\\\\
Множество всех линейных функций обозначают $L$.\\\\
Например: $0,~ 1,~ x,~ \bar x=x\oplus 1,~ x_1\oplus x_2,~ x_1\Leftrightarrow x_2 = x_1\oplus x_2\oplus 1 ~ \in L$\\
	     $ x_1\cdot x_2, ~ x_1\vee x_2 = x_1\cdot x_2 \oplus x_1 \oplus x_2,~ x_1\Rightarrow x2,~ x_1|x_2, x_1 \downarrow x_2 \notin L$\\
\begin{theorem} $11$\\
Класс $L$ замкнут.    
\end{theorem}
\begin{Proof}
    $L=[\{1,~ x,~ x_1\oplus x_2\}]$ – замыкание замыкания = замыканию $\Rightarrow L$ замкнут.
\end{Proof}
\begin{lemma} $3$ (о нелинейной функции)\\
    Если булевая функция нелинейная, то  $x_1\cdot x_2 \in [\{0,~ 1,~ \bar x,~ f\}]$.
\end{lemma}
\begin{Proof}
    Пусть $f = (x_1,\dotso,x_n)$ – нелинейная, тогда по теореме 10 $f$ может быть представлена в виде полинома Жегалкина со степенью $\leqslant 1$. Тогда в представление $G_f(x_1,\dotso ,x_n)$ входит произведение $x_1 \cdot x_2 \Rightarrow$ полином Жегалкина можно представить в следующем виде:\\
    $G_f(x_1,\dotso ,x_n) = x_1\cdot x_2 \cdot p_0(x_3, x_4,\dotso, x_n) \oplus x_1\cdot p_1(x_3, x_4,\dotso, x_n) \oplus x_2\cdot p_2(x_3, x_4,\dotso, x_n) \oplus p_3(x_3, x_4,\dotso, x_n),~~ p_0(x_3, x_4, \dotso, x_n)\not\equiv 0$.\\ 
    $\exists a_3, a_4, \dotso, a_n \in E_2: ~~ p_0(a_3, \dotso, a_n)=1$\\\\
	Пусть $p_1(x_3, x_4,\dotso, x_n)=b_1$,\\
               $p_2(x_3, x_4,\dotso, x_n)=b_2$,\\
            $p_3(x_3, x_4,\dotso, x_n)=b_3$\\\\
	$G_f(x_1, x_2, a_3,\dotso, a_n)=x_1\cdot x_2\oplus b_1\cdot x_1\oplus b_2\cdot x_2\oplus b_3$\\\\
	Сделаем подстановки:\\
 $x_1$ заменим на $x_1\oplus b_2 ~~ \begin{cases}
     x_1, \text{если} b_2 = 0,\\
     \bar x_1, \text{если} b_2 = 1;
 \end{cases}$,\\
 а $x_2$ заменим на $x_2\oplus b_1 ~~ \begin{cases}
     x_2, \text{если} b_1 = 0,\\
     \bar x_2, \text{если} b_1 = 1;
 \end{cases}$.\\ \\
 В результате:\\
	$G_f(x_1\oplus b_2, x_2\oplus b_1, a_3,\dotso, a_n)= (x_1\oplus b_2)( x_2\oplus b_1)\oplus b_ 1(x_1\oplus b_2)\oplus b_2(x_2\oplus b_1)\oplus b_3 = x_1\cdot x_2 \oplus b_1\cdot b_2\oplus b_3,~~ b_1\cdot b_2\oplus b_3 = c$\\
	$x_1\cdot x_2= G_f(x_1\oplus b_2, x_2\oplus b_1, a_3,\dotso, a_n) \oplus c = \begin{cases}
	    G_f, \text{если} c = 0,\\
     \bar G_f, \text{если} c = 1; 
	\end{cases} \Rightarrow x_1\cdot x_2 \in [\{0,~ 1,~ \bar x,~ f\}]$.
\end{Proof}\\\\

Заметим, что классы $T_0,~ T_1,~ S,~ M,~ L$ попарно различны:\\\\
\begin{tabular}{|c|c|c|c|c|c|}
\hline
  & $T_0$ & $T_1$ & $S$ & $M$ & $L$\\
\hline
   0  & + & - & - & + & + \\
   1  & - & + & - & + & + \\
   $\bar x$ & - & - & + & - & + \\
\hline
\end{tabular}\\\\
\begin{theorem} $12$ (Критерий полноты Поста)\\
  Чтобы система функций $A$ была полной необходимо и достаточно, чтобы она целиком не содержалась ни в одном из классов $T_0,~ T_1,~ S,~ M,~ L$. (То есть $f_0,~ f_1,~ f_s,~ f_m,~ f_l\in A$ и $f_0\notin T_0,~ f_1\notin T_1,~ f_m\notin M,~ f_s\notin S,~ f_l\notin L$.)  
\end{theorem}
\begin{Proof}
Необходимость: $A$ -полная и пусть $A\subseteq X$, где $X$ - один из классов $T_0,~ T_1,~ S,~ M,~ L$. Тогда замыкание $[A]\subseteq [X]\notin P_2 \Rightarrow A$ – неполная, из чего следует противоречие.\\
Достаточность: Так как $f_0,~ f_1,~ f_s,~ f_m,~ f_l\in A$ и $f_0\notin T_0,~ f_1\notin T_1,~ f_m\notin M,~ f_s\notin S,~ f_l\notin L \Leftrightarrow f_0(0,\dotso, 0)= 1$. Рассмотрим два случая:\\\\
a)\quad	$f_0(1, \dotso, 1)=1 \Rightarrow f_0(x, \dotso, x) \equiv 1 \in [A]$.\\ С другой стороны, $f_1(1, \dotso, 1)= 0 \Rightarrow f_1(f_0(x,\dotso,x),\dotso, f_0(x,\dotso,x_n))\equiv 0 \in A$. Так как $0, ~ 1 \in [A]$ и $f_m\in [A]$, то по лемме $1$ о немонотонной функции $\bar x \in [0, ~1,~f_m]\in [A]$.\\\\
b)\quad	$f_0(1, \dotso, 1) = 0 \Rightarrow f_0(x, \dotso, x)\equiv \bar x$. По лемме $2$ о не самодвойственной функции: $0,~ 1 \in [\bar x,~ f_s] \subseteq [A] \Rightarrow$ замыканию класса $A$ принадлежат константы и отрицание и по лемме $3$ о нелинейной функции $x_1\cdot x_2\in [0,~ 1,~ \bar x, ~ f_l]\equiv [A]$.\\
Таким образом, $\bar x, ~ x_1 \cdot x_2 \in [f_0,~ f_1, ~f_s, ~ f_m,~ f_l] \subseteq [A]$. По теореме $9$ о сведении А — полная система.  
\end{Proof}\\\\
\textbf{Замечание}: \textit{по теореме Поста можно проверить полноту любой системы из множества булевых функций $A = \{f_1,\dotso, f_t\}$}. Строим таблицу, где строки соответствуют функциям, а столбцы - классам.\\\\
\begin{tabular}{|c|c|c|c|c|c|}
\hline
  & $T_0$ & $T_1$ & $S$ & $M$ & $L$\\
\hline
   $f_1$  &  &  & + &  &  \\
   $f_i$  &  &  & + & - &  \\
   $f_t$  &  &  & + &  &   \\
\hline
\end{tabular}\\\\			
На пересечении строки $f_i$ и столбца записываем: «+», если функция $f_i$ принадлежит классу, записанному в данном столбце, и «-»,  если $f_i$ не принадлежит классу, записанному в данном столбце.\\\\
По теореме Поста, система $A$ является полной тогда и только тогда, когда в любом столбце найдётся хотя бы один минус, и неполной, если есть хотя бы один стобцец, полностью состоящий из плюсов («+»).\\\\
Пример:\\
\begin{tabular}{|c|c|c|c|c|c|}
\hline
  & $T_0$ & $T_1$ & $S$ & $M$ & $L$\\
\hline
   $\bar x$ & - & - & - & + & + \\
   $x \Rightarrow y$ & - & + & - & - & - \\
\hline
\end{tabular} $\Rightarrow$ система полная.
    \end{document}
    