\documentclass[a4paper, 12pt]{report}
\usepackage[T2A]{fontenc} 
\usepackage[utf8]{inputenc}
\usepackage[english,russian]{babel} 
\usepackage{amsmath,amsfonts,amssymb,amsthm,mathtools, tipa}
\usepackage[left=2cm,right=2cm,top=2cm,bottom=2cm,bindingoffset=0cm]{geometry}
\usepackage{ upgreek, mathrsfs, cancel}
\usepackage[unicode]{hyperref}


\newcommand{\RomanNumeralCaps}[1]
    {\MakeUppercase{\romannumeral #1}}

\newenvironment{Proof} % имя окружения для доказательства
{\par\noindent{$\blacklozenge$}} % символ рядом с \begin
{\hfill$\scriptstyle\boxtimes$} % символ рядом с \end

\newenvironment{example} % имя окружения для примеров
{\par\noindent{\textsc{\textbf{Пример}.}}} % символ рядом с \begin
%{\hfill$\scriptstyle\Box$} % символ рядом с \end

\newenvironment{exercise}
{\par\noindent{\textsc{\textbf{Упражнение}.}}}
{\hfill}

\newtheorem*{theorem}{Теорема} % окружение для теорем
\newtheorem*{corollary}{Следствия} % окружение для следствий
\newtheorem*{lemma}{Лемма} % окружение для лемм

\newcommand{\Rm}{\mathbb{R}}
\newcommand{\Cm}{\mathbb{C}}
\newcommand{\I}{\mathbb{I}}
\newcommand{\N}{\mathbb{N}}
\newcommand{\Z}{\mathbb{Z}}
\newcommand{\Q}{\mathbb{Q}}
% команды для множеств 


\title{\textbf{\Huge{Дискретная математика и математическая логика}}\\Конспект по 2 семестру 
	специальностей «экономическая кибернетика» и «компьютерная безопасность»\\(лектор В. И. Бенедиктович)} % оформление титульного листа
\date{} % отключение даты в титульнике
\begin{document}
	\maketitle % отображение титульного листа в документе
	\tableofcontents{}
	\chapter{Булевы функции} % новая глава
	\section*{Замкнутые классы булевых функций}

Пусть $A \subseteq P$\\
$\bullet$ \textbf{Замыканием} $A$ называется множество функций из $P_2$, которые можно выразить в виде формул над $A$ и обозначается $[A]$ .\\\\
Свойства замыкания:
\begin{enumerate}
    \item $A \subseteq [A]$
    \item $A \subseteq B \Rightarrow [A] \subseteq [B]$
    \item $\big[[A]\big] = [A]$
    \item $[A] \cup [B] \subseteq [A \cup B]$
    \end{enumerate}
$\bullet$ 
$A$ - \textbf{полная система} булевой функции, если $[A] = P_2$.\\
$\bullet$ 
Система буевых функций $A$ \textbf{замкнутая}, если $[A] = A$.\\
\begin{example}
$A = \{1, x_1 \oplus x_2\}$ не замкнута, так как $1 \oplus 1 = 0 \notin A$\\
\end{example}

Пусть $A$ - замкнутый неполный класс системы булевых функций. Тогда если $A \subseteq B$, то $B$ - неполная система.\\
\begin{Proof}
$B \subseteq A \Rightarrow [B] \subseteq [A] \neq P_2 \Rightarrow [B] \neq P_2 \Rightarrow B$ - неполная система.
\end{Proof}\\
\begin{center}
    \textbf{Примеры замкнутых классов булевых функций}
    \end{center}
\RomanNumeralCaps{1}) Класс $T_0 = \{f(x_1, \dots , x_n) | f(0, \dots, 0) = 0\}$\\\\
Например:\\
$0,~ x, ~x_1 \cdot x_2, ~x_1 \vee x_2, ~x_1 \oplus x_2 \in T_0$\\
$1, ~\bar x, ~ x_1 \Rightarrow x_2, ~ x_1 | x_2, ~ x_1 \downarrow x_2, ~ x_1 \Leftrightarrow x_2 \notin T_0$\\\\
Мощность класса: $2^n-1$ ненулевых строк $\Rightarrow |T_0| = 2^2-1 = \frac12 2^{2^n}$
\begin{theorem}
Класс $T_0$ замкнут.
\end{theorem}
\begin{Proof}
    Поскольку $x \in T_0$, то достаточно показать, что если $f_1, f_2, \dots, f_n \in T_0$, то $f(f_1, \dots, f_n) \in T_0$. Действительно, $f(f_1(0, \dots, 0), \dots, f_n(0, \dots, 0)) = f(0, \dots, 0)=0$ 
\end{Proof}\\\\
\RomanNumeralCaps{2}) Класс $T_1 = \{f(x_1, \dots, x_n) \in P_2 | f(1, \dots, 1) = 1\}$\\\\
Например:\\
$1, x, x_1 \cdot x_2, x_1 \vee x_2, x_1 \Rightarrow x_2, x_1 \Leftrightarrow x_2 \in T_1$\\
$0, \bar x, x_1|x_2, x_1 \downarrow x_2, x_1 \oplus x_2 \notin T_1$
\begin{theorem}
Класс Класс $T_1$ замкнут.
\end{theorem}
\begin{Proof}
Доказательство аналогично доказательству предыдущей теоремы 
\end{Proof}\\\\
\RomanNumeralCaps{3}) Класс $M$ монотонных функций.\\\\
Введём \textbf{частичный булевый порядок} на $E^n_2$: $\bar \alpha = (\alpha_1, \alpha_2, \dots, \alpha_n)$, $\bar \beta = (\beta_1, \beta_2, \dots, \beta_n) \in E^n_2$\\
Говорят, что $\bar \alpha \leqslant \bar \beta \Leftrightarrow  \alpha_{i} \leqslant \beta_{i}$ для $\forall i$\\\\
$\bullet$ Функция $f(x_1, \dots, x_n)$ называется \textbf{монотонной}, если $\forall \bar \alpha, \bar \beta: \bar \alpha \leqslant \bar \beta \Rightarrow f(\bar \alpha) \leqslant f(\bar \beta)$\\
Множество всех монотонных функций обозначают $M$.\\\\
Например:\\
$0, 1, x, x_1 \cdot x_2, x_1 \vee x_2 \in M$\\
$0, \bar x, x_1 \Rightarrow x_2 \notin M$
\begin{theorem}
Класс $M$ замкнут.
\end{theorem}
\begin{Proof}
Достаточно показать, что если $f_1, f_2, \dots, f_m \in M$, то $f(f_1, \dots, f_m) \in M = \Phi$\\
Пусть $\bar \alpha \leqslant \bar \beta$ , тогда $f_1(\bar \alpha) \leqslant f_1(\bar \beta), \dots, f_m(\bar \alpha) \leqslant f_m(\bar \beta) \Rightarrow (f_1(\bar \alpha), \dots , f_m(\bar \alpha)) \leqslant (f_1(\bar \beta), \dots, f_m(\bar \alpha)) \Rightarrow f(f_1(\bar \alpha), \dots , f_m(\bar \alpha)) \leqslant f(f_1(\bar \beta), \dots, f_m(\bar \alpha))$, то есть $\Phi(\bar \alpha) \leqslant \Phi (\bar \beta)$
\end{Proof}\\
\begin{lemma}
О немонотонной функции\\
Если $f(x_1, \dots, x_n)$ - немонотонная функция, то $\bar x \in [\{0, 1, f\}]$ 
\end{lemma}
\begin{Proof}
Пусть $f(x_1, \dots, x_n)$ - немонотонная функция, то есть $\exists \bar \alpha < \bar \beta \Rightarrow f(\bar \alpha) = 1, f(\bar \beta) = 0 (1 > 0)$. $\bar \alpha < \bar \beta$ означает, что $\exists 1 \leqslant i_1 < i_2 < \dots < i_k \leqslant n$:\\
$\gamma_0 = \bar \alpha = (\alpha_1, \dots, \alpha_{i_1-1}, 0, \alpha_{i_1+1}, \dots, \alpha_{i_{2}-1}, 0, \alpha_{i_2 +1}, \dots, \alpha_{i_k -1}, 0, \alpha_{i_k +1},\dots,  \alpha_n)$  \\
$\gamma_1 = (\alpha_1, \dots, \alpha_{i_1-1}, 1, \alpha_{i_1+1}, \dots, \alpha_{i_{2}-1}, 0, \alpha_{i_2 +1}, \dots, \alpha_{i_k -1}, 0, \alpha_{i_k +1}, \dots,  \alpha_n)$  \\
$\gamma_2 = (\alpha_2, \dots, \alpha_{i_1-1}, 1, \alpha_{i_1+1}, \dots, \alpha_{i_{2}-1}, 1, \alpha_{i_2 +1}, \dots, \alpha_{i_k -1}, 0, \alpha_{i_k +1}, \dots,  \alpha_n)$  \\
$\dots$\\
$\gamma_k = \bar \beta = (\alpha_1, \dots, \alpha_{i_1-1}, 1, \alpha_{i_1+1}, \dots, \alpha_{i_{2}-1}, 1, \alpha_{i_2 +1}, \dots, \alpha_{i_k -1}, 1, \alpha_{i_k +1}, \dots,  \alpha_n)$  \\
$\gamma_0 < \gamma_1 < \gamma_2 < \gamma_3 < \dots < \gamma_k = \bar \beta$\\
Так как $f(\gamma_0) = 1, f(\gamma_k) = 0, f(\gamma_e) = 1, f(\gamma_{e+1}) = 0$, то $\exists l: 0 \leqslant l \leqslant k-1$, то есть $\alpha_e = 0, \beta_e = 1, \forall i \neq l,$  $\alpha_i = \beta_i$\\
Построим функцию $h(x) = f(\alpha_1, \dots, \alpha_{e-1}, x, \alpha_{e+1}, \dots, \alpha_n)$\\
$\begin{cases}
    h(0)  =  f(\bar \alpha) = 1 \\
    h(1) = f(\bar \beta) = 0
  \end{cases}$ $\Rightarrow h(x) \equiv \bar x$
\end{Proof}\\\\\\
\RomanNumeralCaps{4}) Класс $S$ самодвойственных функций.\\\\
$\bullet$ Функция $f^*(x_1, \dots, x_n) = \bar f( \bar x_1, \dots, \bar x_n)$ называется двойственной для функции $f(x_1, \dots, x_n)$\\
$\bullet$ Функция $f(x_1, \dots, x_n)$ называется самодвойственной, если $f(x_1, \dots, x_n) = f^*(x_1, \dots, x_n)$\\
Другими словами:\\
$$\bar f( x_1, \dots, x_n) = f( \bar x_1, \dots, \bar x_n) \eqno (1)$$
$\bullet$ Наборы $\bar \alpha = (\alpha_1, \dots, \alpha_n)$ и $\bar \beta = (\bar \alpha_1, \dots, \bar \alpha_n)$ называются противоположными наборами.\\\\
Например:\\
$x, \bar x \in S$\\
$x_1 \cdot x_2 \notin S$, то есть $(x_1 \cdot x_2)^* = \overline{\overline{x}_1 \cdot \overline{x}_2} = x_1 \vee x_2 \neq x_1 \cdot x_2$
\begin{theorem} 6 \quad
    Класс $S$ замкнут.
\end{theorem}
\begin{Proof}
   Достаточно показать, что $f_1, f_2, \dots, f_n \in S$, то $\Phi = f(f_1, \dots, f_n) \in S$\\
   $\Phi^* (x_1, \dots, x_n) = \bar \Phi (\bar x_1, \dots, \bar x_n) = \bar f(f_1(\bar x_1, \dots, \bar x_n), \dots, f_n(\bar x_1, \dots, \bar x_n))  \stackrel{(1)}{=}$ \\
   $\stackrel{(1)}{=} \bar f(\bar f_1(x_1, \dots, x_n), \dots, \bar f_n(x_1, \dots, x_n)) = f(f_1(x_1, \dots, x_n), \dots, f_n(x_1, \dots, x_n)) = \Phi(x_1, \dots, x_n)$
\end{Proof}
\begin{lemma}
    О несамодвойственной функции.\\
    Если $f(x_1, \dots, x_n)$ - несамодвойственная функция, то $0, 1 \in [\{\bar x, f\}]$
\end{lemma}
\begin{Proof}
    Пусть $f(x_1, \dots, x_n)$ - несамодвойственная функция. Тогда $\exists \bar \alpha = (\alpha_1, \dots, \alpha_n), f(\bar \alpha) = f(\bar \alpha_1, \dots, \bar \alpha_n) = f(\alpha_1, \dots, \alpha_n)$ .\\
    Заменим $\alpha_i$ на $x \oplus \alpha_i$: $\begin{cases}
        x, \text{если} ~ \alpha_i = 0,\\
        \bar x, \text{если} ~ \alpha_i = 1; 
    \end{cases}$\\
    Получим функцию $h(x) \equiv f(x \oplus \alpha_1, \dots, x \oplus \alpha_n)$\\
    $h(0) = f(\alpha_1, \dots, \alpha_n) = c, ~ c = const$\\
    $h(1) = f(\bar \alpha_1, \dots, \bar \alpha_n) = c$\\
    $h(x) = c \Rightarrow \bar c = \bar h(x) \Rightarrow 0, 1 \in [\{\bar x, f\}]$
\end{Proof}
\begin{center}
    \textbf{Полином Жегалкина}
    \end{center}
$\bullet$ Полином Жегалкина — функция вида $\sum\limits_{\{i_1, \dotso, i_k\}\in
\{1, 2, \dotso, n\}} 
a_{i_1,\dotso, a_k} \cdot x_{i_1} 
\cdot \dotso \cdot x_{i_k} \oplus a$ , где $a$ — свободный член.\\\\
Пример: $x_1x_2x_3 \oplus x_1x_2 \oplus x_1x_3 \oplus x_1 \oplus 1$\\
\begin{center}
    Полные системы булевых функций
    \end{center}
    \begin{theorem}7 \quad
    Система функций $A= {x_1\vee x_2, x_1\cdot x_2, \bar x}$ является полной. (Базис Буля)
        \end{theorem}
\begin{Proof}
$f(x_1,\dotso,x_n) \in P_2$. Если булевая функция отлична от нуля, то по следствию из 2 теоремы Шеннона функция $f$ выражается  в виде совершенной дизъюнктивной нормальной формы, в которую входят дизъюнкция, конъюнкция, отрицание, тем самым она принадлежит замыканию класса. Если $f = 0$, то $f = x_1\cdot \bar x_1$.    
\end{Proof}
\begin{theorem} 8 (о сведении)\\
  Если система $A$ — полная и любая функция из $A$ может быть выражена формулой над некоторой системой функций $B$, то $B$ — полная система.  
\end{theorem}
\begin{Proof}
    $[A]=P_2, A \subseteq [B] \Rightarrow P_2 = [A] \subseteq \big[[B]\big] = [B] \subseteq P_2 \Rightarrow [B] = P_2$. То есть $B$ - полная система.
\end{Proof}
\begin{theorem}9\\
 Сдедующие системы являются полными:
 \begin{enumerate}
        \item $A_1 = \{x_1 \vee x_2,~ \bar x\}$
	  \item $A_2 = \{x_1 \cdot x_2,~ \bar x\}$
	  \item $A_3 = \{x_1|x_2\}$
	  \item $A_4 = \{x_1 \cdot x_2, ~x_1\oplus x_2, ~1\}$
  \end{enumerate}
\end{theorem}
\begin{Proof}
\begin{enumerate}
    \item 	По теореме 7 система $\{x_1 \vee x_2,~ \bar x\}$ — полная. По закону де Моргана: $x_1\cdot x_2 = \overline{\overline{x}_1 \vee \overline{x}_2} \Rightarrow x_1\cdot x_2 \in [A_1]$. По теореме 8 следует$[A_1] = P_2$.
\item	По закону де Моргана $x_1 \vee x_2 = \overline{\overline{x}_1 \cdot \overline{x}_2} \Rightarrow x_1\vee x_2 \in [A_2]$. По теореме $8 [A_2] = P_2$.
\item	Можем представить отрицание в виде штриха Шеффера: $\bar x =x|x. x_1 \cdot x_2 = \overline{x_1|x_2} = (x_1|x_2) | (x_1|x_2) \Rightarrow \bar x, ~~ x_1\cdot x_2 \in [A_3]$. По теореме $8$ и доказательству п.$2$ $A_3 = P_2$.
\item	$\bar x = x\oplus 1 \Rightarrow \bar x = [A_4]$. По теореме $8$ и доказательству п.$2$ следует, что $[A_4] = P_2$.
\end{enumerate}
\end{Proof}	
\begin{theorem}$10$ (теорема Жегалкина)\\
Любую булевую функцию $f(x_1, \dotso, x_n)$ можно представить единственным образом в виде полинома Жегалкина $G_f(x_1, \dotso, x_n)$.
\end{theorem}
\begin{Proof}
1) Докажем существование:\\
В силу теоремы $9$ и доказательства п.$4$ система $\{x_1 \cdot x_2, ~x_1\oplus x_2, ~1\}$ полная $\Rightarrow$ любая булевая функция $f(x_1, \dotso, x_n)$ может быть реализована над этой системой. После раскрытия скобок используют дистрибутивность конъюнкции относительно сложения по mod $2$ ($\oplus$) и приведения подобных получаем полином Жегалкина.\\\\
2) Докажем единственность:\\
Подсчитаем количество полиномов Жегалкина от переменных $x_1, \dotso, x_n$. Каждое слагаемое в полиноме Жегалкина имеет вид конъюнкции переменных $x_{i_{1}},\dotso,x_{i_{k}}$ или существует свободный член $1$. Каждая такая конъюнкция определяется подмножеством индексов во множестве индексов $i = \overline{1, n}$ ($\{ i_1, \dotso, i_k\} \subset \{1, \dotso, n\}$). Значит, множество всевозможных слагаемых в полиноме равно количеству подмножеств в $n$-элементном множестве, то есть $2^n$.\\
Чтобы составить полином Жегалкина нужно выбрать подмножество из множества всевозможных слагаемых. Число полиномов Жегалкина равно ${2^2}^n$, что равно количеству булевых функций от $n$ переменных. А так как любая булевая функция имеет полином Жегалкин, представляющий её, то существует единственный полином представляющий булевую функцию.
\end{Proof}\\\\
В силу этой функции полином Жегалкина представляет собой булевую функцию $G_f$.  $G_f(x_1,\dotso,x_n)$ — алгебраически нормальная формула (АНФ) булевой функции. \\\\
Булевая функция $f(x_1,\dotso,x_n)$ существенно зависит от $x_i$ (не является фиктивной переменной) и содержится в каком-либо слагаемом $G_f(x_1,\dotso,x_n)$.\\\\
Пример: $x_1\vee x_2$
\begin{center}
	Методы приведения к виду полинома Жегалкина
 \end{center}
\begin{enumerate}
    \item Метод неопределенных коэффициентов\\ 
$x_1 \vee x_2 = a\cdot x_1x_2 + b\cdot x_1 + c\cdot x_2 +d$; нам необхожимо найти $a, b, c, d$.
Подставим  $(0, 0), (0, 1), (1, 0), (1, 1)$:\\\\
$(0,0)	~~~~~~ d=0$\\
$(0,1)	~~~~~~ 1=c+d \Rightarrow c=1$\\
$(1,0)	~~~~~~ 1=b+d \Rightarrow b=1$\\
$(1,1)	~~~~~~ 1=a+b+c+d \Rightarrow a=1$ (по mod $2$)\\\\
Следовательно, $x_1\vee x_2 = x_1x_2\oplus x_1\oplus x_2$.\\
В общем случае для определения неизвестных коэффициентов при $a_{i_1, \dotso, i_k} x_{i_1}, \dotso, x_{i_k}$ составляется уравнение $G_f(a_1, \dotso, a_i) = f(a_1, \dotso, a_n)$, из чего следует, что всего $2^n$ уравений, $2^n$ неизвестных коэффициентов и в силу теоремы $10$ имеет единственное решение.
\item	Метод эквивалентных преобразований\\
С помощью следующих тождеств: $\bar A=A\oplus 1, ~~ A\vee B = \overline{\overline{A} \cdot \overline{B}} = (A\oplus 1) \cdot (B\oplus 1) \oplus 1 = AB \oplus A \oplus B,~~ A\cdot A=A, ~~ A\cdot 1=A,~~ A\oplus A=0,~~ A\oplus 0=A$, приводим формулу к эквивалентной над системой этих трёх функций ${x_1\cdot x_2, x_1\oplus x_2, \bar x}$ и запишем в виде
 $x_1\vee x_2 = x_1\cdot x_2\oplus x_1\oplus x_2$.
\item Метод треугольника Паскаля \\
Используется, когда функция задана вектором значений. Метод позволяет преобразовать таблицу истинности в полином Жегалкина путем построения вспомогательной треугольной таблицы в соответствии со следующими правилами:
\begin{enumerate}
    \item Строится таблица истинности, в которой строки идут в лексикографическом порядке возрастания двоичных кодов (от $0$ до $1$): $00\dotso 00$, $00\dotso 01$, $00\dotso 10$, $00\dotso 11$, $\dotso$, $11\dotso 11$; 
    \item Строится вспомогательная треугольная таблица, в которой первый столбец совпадает со столбцом значений функции из таблицы истинности;
    \item Ячейка в каждом последующем столбце таблицы получается путем суммирования по mod $2$ двух ячеек: стоящей в той же строке и в строке ниже предыдущего столбца;
    \item Столбцы вспомогательной треугольной таблицы нумеруются двоичными кодами в том же порядке, что и строки таблицы истинности;
    \item Каждому двоичному коду ставится в соответствие один из членов полинома Жегалкина в зависимости от позиций кода, в которых стоят единицы;
    \item Если в верхней строке любого столбца стоит $1$, то соответствующий член входит в полином Жегалкина.
    \end{enumerate}
\begin{tabular}{|c|c|c|}
\hline
$x_1$ & $x_2$ & $x_1 \vee x_2$\\
\hline
   0  & 0 & 0 \\
   0  & 1 & 1\\
   1 & 0 & 1\\
   1 & 1 & 1\\
\hline
\end{tabular}


$\begin{tabular}{c|c|c|c}
 00 & 01 & 10 & 11 \\
\hline
$1$ & $x_1$ & $x_2$ & $x_1 \cdot x_2$\\
\hline
0 & 1 & 1 & 1 \\
1 & 0 & 0\\
1 & 0\\
1\\
\end{tabular}$\\
Из треугольника Паскаля результат: $x_1\vee x_2 = x_1\cdot x_2\oplus x_1\oplus x_2$
\end{enumerate}
V) Класс $L$ линейных функций.
\\\\
$\bullet$ Булевая функция $f(x_1,\dotso,x_n)$ \textbf{линейная}, если она может быть задана в виде полинома Жегалкина степени $\leqslant 1$.
	$$f(x_1,\dotso,x_n)=a_0 \oplus a_1x_1\oplus a_2x_2\oplus \dotso \oplus a_nx_n$$
 где $a_i \in E_2 = \{0, 1\},
 ~~ i= \overline{0, n}$\\\\
Множество всех линейных функций обозначают $L$.\\\\
Например: $0,~ 1,~ x,~ \bar x=x\oplus 1,~ x_1\oplus x_2,~ x_1\Leftrightarrow x_2 = x_1\oplus x_2\oplus 1 ~ \in L$\\
	     $ x_1\cdot x_2, ~ x_1\vee x_2 = x_1\cdot x_2 \oplus x_1 \oplus x_2,~ x_1\Rightarrow x2,~ x_1|x_2, x_1 \downarrow x_2 \notin L$\\
\begin{theorem} $11$\\
Класс $L$ замкнут.    
\end{theorem}
\begin{Proof}
    $L=[\{1,~ x,~ x_1\oplus x_2\}]$ – замыкание замыкания = замыканию $\Rightarrow L$ замкнут.
\end{Proof}
\begin{lemma} $3$ (о нелинейной функции)\\
    Если булевая функция нелинейная, то  $x_1\cdot x_2 \in [\{0,~ 1,~ \bar x,~ f\}]$.
\end{lemma}
\begin{Proof}
    Пусть $f = (x_1,\dotso,x_n)$ – нелинейная, тогда по теореме 10 $f$ может быть представлена в виде полинома Жегалкина со степенью $\leqslant 1$. Тогда в представление $G_f(x_1,\dotso ,x_n)$ входит произведение $x_1 \cdot x_2 \Rightarrow$ полином Жегалкина можно представить в следующем виде:\\
    $G_f(x_1,\dotso ,x_n) = x_1\cdot x_2 \cdot p_0(x_3, x_4,\dotso, x_n) \oplus x_1\cdot p_1(x_3, x_4,\dotso, x_n) \oplus x_2\cdot p_2(x_3, x_4,\dotso, x_n) \oplus p_3(x_3, x_4,\dotso, x_n),~~ p_0(x_3, x_4, \dotso, x_n)\not\equiv 0$.\\ 
    $\exists a_3, a_4, \dotso, a_n \in E_2: ~~ p_0(a_3, \dotso, a_n)=1$\\\\
	Пусть $p_1(x_3, x_4,\dotso, x_n)=b_1$,\\
               $p_2(x_3, x_4,\dotso, x_n)=b_2$,\\
            $p_3(x_3, x_4,\dotso, x_n)=b_3$\\\\
	$G_f(x_1, x_2, a_3,\dotso, a_n)=x_1\cdot x_2\oplus b_1\cdot x_1\oplus b_2\cdot x_2\oplus b_3$\\\\
	Сделаем подстановки:\\
 $x_1$ заменим на $x_1\oplus b_2 ~~ \begin{cases}
     x_1, \text{если} b_2 = 0,\\
     \bar x_1, \text{если} b_2 = 1;
 \end{cases}$,\\
 а $x_2$ заменим на $x_2\oplus b_1 ~~ \begin{cases}
     x_2, \text{если} b_1 = 0,\\
     \bar x_2, \text{если} b_1 = 1;
 \end{cases}$.\\ \\
 В результате:\\
	$G_f(x_1\oplus b_2, x_2\oplus b_1, a_3,\dotso, a_n)= (x_1\oplus b_2)( x_2\oplus b_1)\oplus b_ 1(x_1\oplus b_2)\oplus b_2(x_2\oplus b_1)\oplus b_3 = x_1\cdot x_2 \oplus b_1\cdot b_2\oplus b_3,~~ b_1\cdot b_2\oplus b_3 = c$\\
	$x_1\cdot x_2= G_f(x_1\oplus b_2, x_2\oplus b_1, a_3,\dotso, a_n) \oplus c = \begin{cases}
	    G_f, \text{если} c = 0,\\
     \bar G_f, \text{если} c = 1; 
	\end{cases} \Rightarrow x_1\cdot x_2 \in [\{0,~ 1,~ \bar x,~ f\}]$.
\end{Proof}\\\\

Заметим, что классы $T_0,~ T_1,~ S,~ M,~ L$ попарно различны:\\\\
\begin{tabular}{|c|c|c|c|c|c|}
\hline
  & $T_0$ & $T_1$ & $S$ & $M$ & $L$\\
\hline
   0  & + & - & - & + & + \\
   1  & - & + & - & + & + \\
   $\bar x$ & - & - & + & - & + \\
\hline
\end{tabular}\\\\
\begin{theorem} $12$ (Критерий полноты Поста)\\
  Чтобы система функций $A$ была полной необходимо и достаточно, чтобы она целиком не содержалась ни в одном из классов $T_0,~ T_1,~ S,~ M,~ L$. (То есть $f_0,~ f_1,~ f_s,~ f_m,~ f_l\in A$ и $f_0\notin T_0,~ f_1\notin T_1,~ f_m\notin M,~ f_s\notin S,~ f_l\notin L$.)  
\end{theorem}
\begin{Proof}
Необходимость: $A$ -полная и пусть $A\subseteq X$, где $X$ - один из классов $T_0,~ T_1,~ S,~ M,~ L$. Тогда замыкание $[A]\subseteq [X]\notin P_2 \Rightarrow A$ – неполная, из чего следует противоречие.\\
Достаточность: Так как $f_0,~ f_1,~ f_s,~ f_m,~ f_l\in A$ и $f_0\notin T_0,~ f_1\notin T_1,~ f_m\notin M,~ f_s\notin S,~ f_l\notin L \Leftrightarrow f_0(0,\dotso, 0)= 1$. Рассмотрим два случая:\\\\
a)\quad	$f_0(1, \dotso, 1)=1 \Rightarrow f_0(x, \dotso, x) \equiv 1 \in [A]$.\\ С другой стороны, $f_1(1, \dotso, 1)= 0 \Rightarrow f_1(f_0(x,\dotso,x),\dotso, f_0(x,\dotso,x_n))\equiv 0 \in A$. Так как $0, ~ 1 \in [A]$ и $f_m\in [A]$, то по лемме $1$ о немонотонной функции $\bar x \in [0, ~1,~f_m]\in [A]$.\\\\
b)\quad	$f_0(1, \dotso, 1) = 0 \Rightarrow f_0(x, \dotso, x)\equiv \bar x$. По лемме $2$ о не самодвойственной функции: $0,~ 1 \in [\bar x,~ f_s] \subseteq [A] \Rightarrow$ замыканию класса $A$ принадлежат константы и отрицание и по лемме $3$ о нелинейной функции $x_1\cdot x_2\in [0,~ 1,~ \bar x, ~ f_l]\equiv [A]$.\\
Таким образом, $\bar x, ~ x_1 \cdot x_2 \in [f_0,~ f_1, ~f_s, ~ f_m,~ f_l] \subseteq [A]$. По теореме $9$ о сведении А — полная система.  
\end{Proof}\\\\
\textbf{Замечание}: \textit{по теореме Поста можно проверить полноту любой системы из множества булевых функций $A = \{f_1,\dotso, f_t\}$}. Строим таблицу, где строки соответствуют функциям, а столбцы - классам.\\\\
\begin{tabular}{|c|c|c|c|c|c|}
\hline
  & $T_0$ & $T_1$ & $S$ & $M$ & $L$\\
\hline
   $f_1$  &  &  & + &  &  \\
   $f_i$  &  &  & + & - &  \\
   $f_t$  &  &  & + &  &   \\
\hline
\end{tabular}\\\\			
На пересечении строки $f_i$ и столбца записываем: «+», если функция $f_i$ принадлежит классу, записанному в данном столбце, и «-»,  если $f_i$ не принадлежит классу, записанному в данном столбце.\\\\
По теореме Поста, система $A$ является полной тогда и только тогда, когда в любом столбце найдётся хотя бы один минус, и неполной, если есть хотя бы один стобцец, полностью состоящий из плюсов («+»).\\\\
Пример:\\
\begin{tabular}{|c|c|c|c|c|c|}
\hline
  & $T_0$ & $T_1$ & $S$ & $M$ & $L$\\
\hline
   $\bar x$ & - & - & - & + & + \\
   $x \Rightarrow y$ & - & + & - & - & - \\
\hline
\end{tabular} $\Rightarrow$ система полная.


\begin{center}
	\textbf{Минимизация булевых функций}
\end{center}
Элементарная конъюнкция — выражение вида $K=x_{i_1}^{\sigma_1}, \dotso , x_{i_r}^{\sigma_r}$, 
где $r$ — ранг конъюнкции, $i_k\in \{1,\dotso,n\}$, $\sigma_i\in\{0, 1\}$, $i_j\neq i_k$ при $j\neq k$.\\\\
$ x_{i_j}^{\sigma_j} = \begin{cases}
	x_{i_j}, ~ \sigma_j = 1,\\
	\bar x_{i_j},  ~ \sigma_j = 0  
\end{cases}$ — литералы.\\\\
Утверждение: const $= 1$ - элементарная конъюнкция, $r=0$.\\\\
$\bullet$ Полная элементарная конъюнкция — элементарная конъюнкция, в которой каждая переменная $ f_1, \dotso, f_n $ входит в нее не более 1 раза вместе с отрицанием.\\\\
$\bullet$ Дизъюнктивная нормальная форма (ДНФ) — $ R = \forall_{i=1}^s K_i $, где $K_i \neq K_j$ при $ i\neq j$, $K_i$ - элементарные коньюнкции.\\\\
$ \bullet $ Совершенная ДНФ (СДНФ) — ДНФ, состоящая из различных полных элементарных конъюнкций.\\\\
Булевая функция может быть представлена в виде ДНФ не единственным образом.\\\\
СДНФ и СКНФ обеспечивают единственность представления любой булевой функции, но они неудобны при технической реализации булевой функции, поэтому их преобразуют в наиболее простые формы, более рациональные с точки зрения их реализации.\\
Вводят индекс простоты, характеризующий сложность ДНФ. В качестве индекса используют количество переменных и их отрицаний, их литералов.\\\\
$\bullet$
Минимальная ДНФ — ДНФ, содержащая наименьшее количество литералов среди всех ДНФ, реализующих данную булевую функцию.\\\\
\textit{Замечание}:\\
Число различных элементарных конъюнкций от $ n $ переменных равно $ 3^n $, так как любая переменная может входить в конъюнкцию, не входить в конъюнкцию или входить с отрицанием, следовательно количество ДНФ равно $ 2^{3^n} $.\\\\
$\bullet$
Импликанта булевой функции $f(x_1,\dotso,x_n)$ — булевая функция $g(x_1,\dotso,x_n)$, если для любого набора $\bar \alpha \in E^n_2$ из $g(\bar \alpha)=1$, следует, что $f(\bar \alpha)=1$ или $g\vee f \equiv f$.\\
$f$ — имплицента булевой функции $g(x_1,\dotso,x_n)$.\\\\
\textit{Замечание}:\\
Всякая элементарная конъюнкция, входящая в булевую функцию является её импликантой.\\
Конъюнкция любого числа импликант также импликанта болевой функции.\\\\
$\bullet$
Импликанта от $f(x_1,\dotso,x_n)$, являющаяся элементарной конъюнкцией, называется простой, если никакая её часть не является булевой функцией $f$.\\\\
$\bullet$
Сокращенная ДНФ — ДНФ, реализующая $f$ и состаящая из всех простых импликант. \\\\
$\bullet$
Тупиковая сокращенная ДНФ — сокращенная ДНФ, если отбрасывание любых элементов конъюнкции или литерала приводит ее к неэквивалентной ДНФ.\\\\
$\bullet$
Кратчайшая ДНФ — ДНФ, содержащая минимальное количество импликант.\\\\
\textit{Замечание}:\\
Булевая функция может иметь несколько тупиковых ДНФ.\\\\
$\bullet$
Минимальная ДНФ — тупиковая ДНФ f, содержащая наименьшее количество литералов.\\\\
\textit{Замечание}:\\
Булевая функция может иметь несколько минимальных и кратчайших ДНФ; существует тупиковая ДНФ, не являющаяся кратчайшей, и существует кратчайшая ДНФ, не являющаяся минимальной.\\\\
\textit{Утверждение}:\\
Минимальная ДНФ булевой функции $f(x_1,\dotso,x_n)$ получается из сокращенной ДНФ путем удаления некоторых элементарных конъюнкций.\\
\begin{Proof}
	Покажем, что все импликанты, составляющие минимальную ДНФ, являются простыми.\\
	От противного:\\
	Пусть $R=k_1\vee k$, где $k_1$ -не простая конъюнкция, $k$ - дизъюнкция остальных конъюнкций, которые входят в разложение булевой функции.\\
	Так как $k_1$ - не простая конъюнкция, то её можно представить в виде произведение других конъюнкций: $k_1 = k_1' \cdot k_2''$, где $k_1'$ - импликанта $f$, то есть $k_1' \vee f = f$. Тогда $f = (k_1\vee k)\vee k_1' = ( k_1' \cdot k_2'' \vee k)\vee k_1' = k_1'k_2'' \vee k k_1' = k_1' \vee k$, что меньше, чем $k_1'\cdot k_2'' ~ \Rightarrow R$ не минимальная, получили противоречие.
\end{Proof}\\
Этапы минимизации булевой функции:
\begin{enumerate}
	\item Построение СДНФ;
	\item Получение сокращенной ДНФ;
	\item Нахождение всех тупиковых ДНФ;
	\item Выбор из тупиковых ДНФ минимальных;
\end{enumerate}
\begin{theorem} $1$ (Квайна)\\
	Если в произвольной ДНФ булевой функции произвести всевозможные обобщенные склеивания, а затем выполнить все поглощения, то получится сокращенная ДНФ.
\end{theorem}
I) \quad Метод Блейка-Порецкого:
\begin{enumerate}
	\item Построение СДНФ;
	\item По теореме Квайна, производим все обобщённые склеивания, пока возможно, по правилу:\\
	$xk_1 \vee \bar x k_2 = xk_1 \vee \bar x k_2\vee k_1k_2$ 
	\item По теореме Квайна производим всевозможные поглопоглощения по правилу:\\
	$k_1 \vee k_1k_2 = k_1$
	\item удаляем лишние конъюнкции по правилу обобщённого склеивания:\\
	$xk_1 \vee \bar x k_2 \vee k_1k_2 = xk_1 \vee \bar x k_2$
\end{enumerate}
Пример:\\
$\omega(f)=(11011011)$\\
\begin{tabular}{|c|c|c|c|}
	\hline
	$x$ & $y$ & $z$ & $f(x, y, z)$ \\
	\hline
	$0$ & $0$ & $0$ & $1$ \\
	$0$ & $0$ & $1$ & $1$ \\
	$0$ & $1$ & $0$ & $0$ \\
	$0$ & $1$ & $1$ & $1$ \\
	$1$ & $0$ & $0$ & $1$ \\
	$1$ & $0$ & $1$ & $0$ \\
	$1$ & $1$ & $0$ & $1$ \\
	$1$ & $1$ & $1$ & $1$ \\
	\hline
\end{tabular}\\
\begin{enumerate}
	\item $f(x, y, z) = \bar x \bar y \bar z \vee \bar x \bar y z \vee \bar x y z \vee x \bar y \bar z \vee x y \bar z \vee x y z$
	\item $ \bar x \bar y \bar z \vee \bar x \bar y z = \bar x \bar y \bar z \vee \bar x \bar y z \vee \bar x \bar y$\\
	$\bar x \bar y z \vee \bar x y z = \bar x \bar y z \vee \bar x y z \vee \bar x z$\\
	$\bar x \bar y \bar z \vee x \bar y \bar z = \bar x \bar y \bar z \vee x \bar y \bar z \vee \bar y \bar z $\\
	$x y \bar z \vee x y z = x y \bar z \vee x y z \vee x y$\\
	$\bar x \bar y \bar z \vee \bar x \bar y z \vee \bar x y z \vee x \bar y \bar z \vee x y \bar z \vee x y z \vee \bar x \bar y \vee \bar x z \bar y \bar z \vee x y$
	\item Поглощаем:\\
	$\bar x \bar y z \vee \bar x \bar y z \vee \bar x \bar y = \bar x \bar y$\\
	$x \bar y \bar z \vee \bar y \bar z = \bar y \bar z$\\
	$\bar x y z \vee \bar x z = \bar x z$\\
	$x y z \vee x y z \vee x y = x y$\\
	$f(x, y, z) = \bar x \bar y \vee \bar x z \vee \bar y \bar z \vee x y$
	\item Склеиваем:\\
	$f(x, y, z) =  \bar x z \vee \bar y \bar z \vee x y$
\end{enumerate}

II) \quad Метод Квайна:\\
2 операции:\\\\
а) попарное неполное склеивание:\\
$kx \vee k \bar x = kx \vee k \bar x \vee k$\\\\
б) элементарное поглощение:\\
$kx^{\sigma} \vee k = k$\\
\begin{theorem} $2$ (Квайна)\\
	Исходя из СДНФ, если произвести всевозможные операции а) и б), то получим сокращенную ДНФ   
\end{theorem}
Алгоритм Квайна:
\begin{enumerate}
	\item По таблице истинности стротм СДНФ;
	\item Выполняем всевозможные операции неполного попарного склеивания для элементарных конъюнкций длины $n$
	\item Выполняем всевозможные операции элементарного поглощения для элементарных конъюнкций длины $n-1$
	\item В результате получится множество элементарных конъюнкций, состоящее из 2 подмножеств: элементарных конъюнкций длины $n$ и элементарных конъюнкций длины $n-1$
	\item Если множество элементарных конъюнкций длины $n-1$ не пусто, то заново выполняем операции а) и б)
	\item Завершается алгоритм, когда подмножество элементарных конъюнкций не будет либо пустым, либо нельзя будет выполнить ни одной операции. С результате получим сокращённую ДНФ
\end{enumerate}
Переход от СДНФ к сокращенной ДНФ происходит с помощью импликантной матрицы Квайна. В этой матрице полные элементарные конъюнкции СДНФ записываются в заголовке столбцов, а простые импликанты сокращенных ДНФ в заголовках строк. На пересечении ставится «+», если импликант в строке входит в конъюнкцию $k_j$.\\
Минимальные ДНФ строятся по этой матрице:\\
Вначале строится тупиковая ДНФ, в которой выбирается минимальное число простых импликант сокращенной ДНФ, дизъюнкция которых накроет плюсами все столбцы импликантной матрицы, т.е. каждый столбец содержит «+», стоящий на пересечении со строкой, соответствующей одной из выбранных импликант.\\
Далее из тупиковых ДНФ выбирается минимальная ДНФ, имеющая наименьшее число вхождений переменных из всех построенных тупиковых из матрицы.\\\\
\textit{Замечание}:\\
Для столбцов, имеющих только один «+», соответствующие им простые импликанты сокращенной ДНФ являются базисными, дизъюнкции которых составляют ядро булевой функции, которое обязательно входит в минимальную ДНФ.\\\\
Пример:
\begin{enumerate}
	\item $f(x, y, z) = \bar x \bar y \bar z \vee \bar x \bar y z \vee \bar x y z \vee x \bar y \bar z \vee x y \bar z \vee x y z$
	\item Попарное неполное склеивание:\\
	$\bar x \bar y \bar z \vee \bar x \bar y z \vee \bar x y z \vee x \bar y \bar z \vee x y \bar z \vee x y z \vee \bar x \bar y \vee \bar y \bar z \vee \bar x z \vee y z \vee x \bar z \vee x y$\\
	\item Сокращённая ДНФ:\\
	$f(x, y, z) = \bar x \bar y \vee \bar y \bar z \vee \bar x z \vee y z \vee x \bar z \vee x y$
\end{enumerate}
\begin{tabular}{|c|c|c|c|c|c|c|c|}
	\hline
	& $\bar x \bar y \bar z$ & $\bar x \bar y z$ & $\bar x y z$ & $x \bar y \bar z$ & $x y \bar z$ & $x y z$ &\\
	\hline
	$\bar x \bar y$ & $\oplus$&$\oplus$& & & & & V\\
	\hline
	$\bar y \bar z$ & $\pm$ &  &  & $\pm$ &  &  & W\\
	\hline
	$\bar x z$ &  & $\pm$ & $\pm$ &  &  &  & W\\ 
	\hline
	$ y z$ &  &  & $\oplus$ &  &  & $\oplus$ & V\\
	\hline
	$x \bar z$ &  &  &  & $\oplus$ & $\oplus$ &  & V\\
	\hline
	$x y$ &  &  &  &  & $\pm$ &  $\pm$ & W\\
	\hline
\end{tabular}\\\\
Тупиковые ДНФ:\\
$\bar x \bar y \vee y z \vee x \bar z$ и\\
$y \bar z \vee \bar x z \vee x y$








\chapter{Теория графов} % новая глава
\section*{Основные понятия}

$\bullet$ \textbf{Граф} - следующая упорядоченная пара: $G = (V, E)$, где $V$ - непустое множество, состоящая из вершин графа, а $E \subseteq V^{(2)}$, где $V^{(2)}$ - все двухэлементные подмножества из $V$.\\\\
$V^{(2)} = \{ \{v, w \} | v, w \in V \}$\\\\

$\bullet$ Граф \textbf{конечный}, если множество вершин конечно$|V| = n$. \\\\
Если $|E|=n$, то граф $G$ обозначается $G(n,m)$ \\\\
$n=|V|$ - порядок графа $G$.\\
$m=|E|$ - размер графа $G$.\\
Если $n=1$, то граф тривиальный. \\
$\bullet$ Граф \textbf{простой}, если он не имеет петель - $\{v, w\}$ и не имеет кратных ребер, то есть несколько пар $\{v, w\}$.\\\\
Вершины $v$ и $w$ \textbf{смежные}, обозначают $v\tilde w$, если ребро $\{v, w\} \in E$.\\\\
Ребро $e=\{v, w\}=vw$ инцидентно вершинам $v$ и $w$, а эти вершины $v,~w$ называют концами $e$.\\\\
Два ребра называют смежными, если существует $ v \in V $, которой они инцидентны или существует их общий конец.\\\\
Пусть есть $v \in V(G)$. тогда множество всех  вершин $u$ таких, что $u \tilde v, ~~ N_G(v)= \{u \in V(G)|u\tilde v\}$ называют окружением вершины $v$.\\\\
$N_G(v) \cup \{ v \} = N_G [ v ] = N[v]$  — замкнутое окружение вершины $v$. \\\\

$\bullet$ \textbf{Степень вершины} $ v ~~~~ deg_G(v) = |N_G(v)|$ равна числу рёбер, выходящих из данной вершины.\\\\

Если $V' \subseteq V(G)$, то окружение множества вершин $V'$~ - множество $N_G(V') = \underset{v \in V'}{\cup} N_G(v)$\\\\

$\bullet$ Вершина, которая смежна с любой вершиной из $V$ называется \textbf{доминирующей}, то есть $v \in V;~ \forall u \in V \setminus \{v\}, ~ v\sim u $\\\\

Если $v \in V$, такая что для любой $u \in V \setminus \{v\}$ $v \sim u$, то $v$ - доминирующая.

$ \bullet $ \textbf{Доминирующее множество} ~ - множество $U$, такое что для любого $v \in V(G) \setminus \{u\}$ существует $u \in U$, такое что $v \sim u$.

$ \bullet $ \textbf{Число доминирования} $\gamma(G)$~- мощность наименьшего доминирующего множества.\\

$ \bullet $ Если $\deg(v) = 0$, то $v$~-\textbf{ изолированная}.\\

$ \bullet $ Если $\deg(v) = 1$, то $v$~- \textbf{висячая} или \textbf{лист}.\\

Минимальная степень вершин~ - $\delta(G) \geq 0$.\\

Максимальная степень вершин~ - $\Delta(G) \leq n - 1$.\\

$ \bullet $ Список степеней, упорядоченный по возрастанию ~- \textbf{\textbf{степенная последовательность}} $\delta = d_1 \leq d_2 \leq \dots \leq d_n \leq \Delta$.

\textbf{Регулярный/однородный граф} степени $k$~($k$-регулярный)~- граф, такой что $\deg(v) = k$, $v \in V$. При $k = 3$ граф кубический.\\\\

$ \bullet $ \textbf{Псевдограф} — граф, который может содержать петли и кратные ребра, кратные петли.\\\\
$ \bullet $ Мультиграф — граф, который может содержать кратные ребра.\\\\
$ \bullet $ Если $E⊂V^2$ (составлено из упорядоченных двоек/пар), то граф \textbf{G—ориентированный} (орграф), а его ребра — дуги.\\\\
Если $ v $ — начало дуги $(v,w)$, то дуга исходит из $v$.\\ 
Если $ w  $ — конец дуги $(v,w)$, то дуга заходит в $ w $.\\\\

$ \bullet $ Количество заходящих дуг — \textbf{полустепень захода} $ deg - w $. \\

$ \bullet $ Количество исходящих дуг — \textbf{полустепень исхода} $ deg + v $.\\
Если в $ E $ существует состоящие из более чем двух вершин $e\{ v,~ w,~ u\}$, то $G$ —\textbf{гиперграф}.\\\\
Для любого(мульти/псевдо) графа справедлива лемма:
\begin{lemma} Лемма о рукопожатиях:\\
$ deg(v_1) + deg(v_2)+ \dotso + deg(v_n) = 2m, ~~~ \forall i,~~ v_i\in V.  $
\end{lemma}
\begin{Proof}
Это следует из того, что вклад каждого ребра в левую часть такой же, что и петли в правую часть равенства, равный двум.
\end{Proof}

\textbf{Следствие}:\\
 Количество вершин нечетных степеней четно.\\\\
$On$ — пустой граф — граф, состоящий из $ n $ изолированных вершин (нет ребер).\\\\
$ Kn $ — полный граф — граф, где все вершины попарно смежны: $m=C_n^2$.\\\\
$ Pn $ —цепь на $ n $ вершинах — граф, у которого $2$ листа, а остальные вершины имеют степень $2$. Длина (см. далее) равна $n-1$.\\\\
$ Cn $ —цикл на $ n $ вершинах — связный (см. далее) граф, у которого все вершины имеют степень $ 2 $.
$G'=(V',E')$ — подграф графа $G$, если $V'\subseteq V,~~ E\subseteq E'$.\\\\
$ \bullet $ \textbf{Собственный подграф} — граф $G'$, где $V'⊂V$ и/или $E⊂E'$.\\\\
Если $V'⊂V(G)$, то \textbf{порожденный (индуцированный) подграф} множества вершин $V' G[V'] =def(V',E')$, где $E'={vu \in E(G) | v, u \in V'}$.\\
$W\subseteq V(G) \Rightarrow G[V\setminus W]=def G-W$\\
$F\subseteq E(G) \Rightarrow G+F = (V, E(G)\cup F) \Rightarrow G-F = (V, E(G)\setminus F)$\\
$V=\{v\} \Rightarrow G - \{v\}$ аналогично $G - v$\\
$F=\{e\} \Rightarrow G+\{e\}$ аналогично $G + e$\\
$\Rightarrow G-\{e\}$ аналогично $G - e$\\\\

Независимое множество вершин $W\subseteq V(G)$ — $W$, такое что $G[W]$ - пустой. \\
Мощность максимального $ W $ — число независимостей $ \alpha(G) $.\\
Клика $W\subseteq V(G)$ — $W$, такое что $G[W]$ - полный.\\
Мощность максимального $W$ — кликовое число $w(G)$.\\
$ \bullet $ \textbf{Остов (субграф)} — подграф $G'=(V',E')$, такой что $V'=V(G)$.\\\\
Пусть $G=(V,E)$ — произвольный подграф.\\
Реберный граф — $L(G)$, такой что:\\
1) вершины $L$ — ребра $G$\\
2) $2$ вершины $ ij $ и $ kl $ принадлежащие $ E(G) $ — смежные, если в $ G~~ ij $ и $ kl $ — смежные.\\
$ deg(ij) $ в $ L(G)=deg(i) $ в $G + deg(j) $ в $ G – 2 $.\\
\textbf{Замечание}: порядок $  |L(G)|=m$, если $G(n,m) $.\\\\
\textbf{Утверждение}: размер $ L(G)=m_L=\frac12 \cdot \sum\limits_{i=1}^n d_i^2  - m $.\\
\begin{Proof}
 Вершина $ i $ в $ G $ имеет степень $ d_i $, то в $ L(G) $ она образует $ C_{d_i}^2 $ ребер, каждая пара ребер—вершина в$  L(G) \Rightarrow m_L=\sum\limits_{i=1}^n  C_{d_i}^2  = \frac12 \cdot \sum\limits_{i=1}^n d_i (d_i - 1) $=[по дистрибутивности и лемме о рукопожатиях] =$\frac12 \cdot \sum\limits_{i=1}^n d_i^2  - m $.
\end{Proof}

\textbf{Следствие}:$ \sum\limits_{i=1}^n d_i^2 = \sum \limits_{ij\in E(G)} (d_i+d_j ) $  — первый индекс Загреба
\begin{Proof}
$ 2m_L=\sum \limits_{ij\in E(G)} d_ij =\sum \limits_{ij\in E(G)} (d_i + d_j - 2) =\sum \limits_{ij\in E(G)} (d_i + d_j ) - 2m $;\\
$ 2(m_L+m)= \sum \limits_{ij\in E(G)} (d_i+d_j ) $  и $ 2(m_L+m)= \sum \limits_{i=1}^2 d_i^2 $ .
\end{Proof}\\

$ \bullet $ \textbf{Помеченный граф} — 1) граф, вершинам или ребрам которого присвоены какие-либо метки (числа, буквы); 2)граф порядка $ n $, если его вершинам присвоить попарно различные номера от $ 1 $ до $ n $.\\
\begin{theorem} $1$\\
 Количество помеченных графов порядка $ n = 2^{C_n^2 } $.
\end{theorem}
\begin{Proof}
 По определению количество ребер в полном графе порядка $ n $ равно числу всевозможных пар вершин равное $ C_n^2 \Rightarrow $ количество всех графов с фиксированным множеством вершин равное числу всех подмножеств множества всевозможных пар вершин = $ 2^{C_n^2} $.
\end{Proof}




    \end{document}
    